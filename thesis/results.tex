\chapter{Experimental Results}
\label{chap:results}

\newcommand{\plot}[1]{
  \subfigure{
    \includegraphics[width=0.5\textwidth,angle=-90]{data/all/#1}
  }

  \subfigure{
    \includegraphics[width=0.5\textwidth,angle=-90]{data/random/#1}
  }

  \subfigure{
    \includegraphics[width=0.5\textwidth,angle=-90]{data/nonrandom/#1}
  }
}

\section{Setup}

We will now evaluate STRUCTure on DAS-4 (\url{http://www.cs.vu.nl/das4/})
cluster at VU University Amsterdam (\url{http://www.few.vu.nl})
which is composed of 74 machines with 24GiB of memory and with
dual quad core Intel E5620 processors running at 2.4GHz. The
cluster allows evaluation of different types of parallelization:
\begin{inparaenum}[1)]
  \item when more nodes are added and
  \item when more CPUs per node are added.
\end{inparaenum}

The nodes have hyper threading enabled so each of them has \emph{16
logical cores}. A pair of logical cores on the same physical core
share main execution resources, but not the architectural state
(e.g. control and general purpose registers).  Except where otherwise
indicated tests will run on \emph{one node} with \emph{16 threads}
(to employ all 16 logical cores). Constellation does not spawn more
than specified number of threads. However, Java Virtual Machine will
use the additional computing power for things like garbage collector.

STRUCTure was evaluated using 678 problems
of medium difficulty from SAT Competition 2009
(\url{http://www.satcompetition.org/2009/}). The instances are
divided into three categories:
\begin{itemize}
  \item formulas in \emph{application} category encode different
  real-life problems and are typically very large (up to tens of
  millions of variables and clauses);
  \item \emph{crafted} SAT instances are generally small and designed
  to stress solvers;
  \item \emph{random} instances contain random $k$-SAT formulas.
\end{itemize}

In the following tests, instances were divided into random and
nonrandom (application and crafted) categories. There are 380 random
instances and 298 nonrandom instances.

On the tests with a large time limit of 45 minutes all instances were
measured once to conserve computational resources. On all other tests
the time limit was 15 minutes and time measured is the average of
three runs.  If for STRUCTure timed out on any run of an instance
that instance is not used for the graphs.  \footnote{The extra
time gives the illusion that STRUCTure does better in 15 minutes on
tests with larger time limit than on tests with smaller time limit.
In reality, because the performance varies, instances which require
a running time close to the upper time limit might or might not
be included in the graphs.} The time limit in SAT Competitions is
usually 20 minutes, but the rules of DAS-4 cluster allow only 15
minutes jobs during the daytime.  With extra 5 minutes STRUCTure
can solve only few extra instances so the additional time is not
required to understand performance a scalability.

With a time limit of 15 minutes, STRUCTure can solve 197 instances
on a single DAS-4 node, but state of the art SAT solvers solve up
to 350.


\section{Failed Literal Probing}

We continue next to investigate the influence of Failed
Literal Probing (FLP, see Section \ref{sssec:flp}) on overall
performance. In Figure \ref{fig:flp} the graph of solved instances
over time is plotted for all, random and nonrandom instances with
varying number of probes.

On random $k$-SAT instances FLP works best with high number of
probed literals. Since FLP implementation uses a weak form of BCP
it is important that instances are simplified so that they contain
at least some binaries.  On non-random instances FLP performs best
with fewer probes, i.e. only 32.  Unlike on random instances FLP
works worse when number of probes is increased.  Overall probing
128 variables gives best results, but the performance does not vary
much with different number of probes.

\begin{figure}
  \centering
  \plot{flp}
  \caption{Performance when varying number literal probes.}
  \label{fig:flp}
\end{figure}


\section{Performance of Reasoning Procedures}
\label{sec:perf-reasoning}

An important part of any sat solver is the simplification
procedures and reasoning techniques it performs. The algorithms
implemented in STRUCTure are described in Chapter \ref{chap:sat},
\nameref{chap:sat}, and subsection \ref{ssec:learning},
\nameref{ssec:learning}.

Before continuing with the test results, we make an important
observation: there is a huge overlap between different techniques
(e.g. BCE implies PL and DVR \cite{Jarvisalo_blockedclause}),
but they are performed together because some are faster or perform
better in different scenarios.

Figure \ref{fig:disable} gives the graph of solved instances
over time for all, random and nonrandom instances with different
simplification procedures disabled.  The default line represents
behaviour when \emph{all} reasoning procedures are enabled. The
other lines represent the behaviour when one reasoning technique
is disabled. Similarly, table \ref{tbl:disable} lists which
simplification procedures improve performance.

The simplification procedures are divided in two categories
depending when they are performed:
\begin{itemize}
  \item before search, in Simplification component:
  DVR (\ref{ssec:dvr}), BCE (\ref{ssec:bce}) and VE (\ref{ssec:ve}); and
  \item during search, in Searching component:
  Split (\ref{sssec:split}), learning (\ref{ssec:learning}),
  and SSS (\ref{ssec:sss}) and HUR (\ref{ssec:hbr}) in Solve Activity.
\end{itemize}

\begin{figure}
  \centering
  \plot{disable}
  \caption{Performance with different simplification procedures disabled}
  \label{fig:disable}
\end{figure}

\begin{table}
  \centering
  \begin{tabular}{| c | c | c | c |}
    \hline
    \multicolumn{4}{|c|}{\textbf{Simplification component (before search)}} \\
    \hline
    & \textbf{all} & \textbf{random} & \textbf{nonrandom} \\
    \hline
    Blocked Clause Elimination & - & - & = (?) \\
    Variable Elimination & + & = & + \\
    Dependent Variable Elimination & - & = & - \\
    \hline
    \hline
    \multicolumn{4}{|c|}{\textbf{Searching component (during search)}} \\
    \hline
    & \textbf{all} & \textbf{random} & \textbf{nonrandom} \\
    \hline
    Split & - & - & - \\
    Learning & ++ & ++ & ++ \\
    Self-Subsumption (in Solve activity) & - & - & + \\
    Hyper Unit Resolution (in Solve activity) & + & = & ++ \\
    \hline
  \end{tabular}

  \caption{Effect on performance of different simplification procedures.
  - bad, = no change, + good, ++ extremely good.}
  \label{tbl:disable}
\end{table}

First, from all reasoning techniques, learning has the biggest
performance impact. The gain is larger on random instances than on
nonrandom instances, though learning was expected to benefit more
industrial instances \cite{DBLP:series/faia/SilvaLM09}. Section
below shows that STRUCTure does not learn many clauses which are
required for bigger industrial applications.

At the other extreme we notice that trying to split instances does not
provide any real benefit and, therefore, Split activity should be disabled
on most problem types.

Before search, during the simplification phase, VE improves
performance on nonrandom instances and has almost no impact on random
instances. STRUCTure performs worse with DVR enabled. BCE gives
worse performance on random instances, but on nonrandom instances
the performance is not affected.  The performance of combinations
of BCE and VE is given in appendix \ref{chap:bce-ve}.

During search, both HUR and SSS are useful for nonrandom instances
with HUR being almost as important as learning. On random
instances STRUCTure performs better without these two techniques
enabled. Overall it is better to disable SSS during search, but
leave HUR enabled.

We conclude that random instances benefit only from searching and
learning, while most reasoning algorithms improve running time
on nonrandom instances.  Testing other combinations of reasoning
procedures is future work.



\section{Solved Instances and Depth Histogram}
\label{sec:histograms}

For this section we picked a satisfiable industrial instance,
\textsf{grieu/vmpc\_28.cnf}, and measured how STRUCTure behaves
internally with respect to processed instances. The selected
instance has 784 variables, 108,080 clauses, 391,300 literals in all
clauses \footnote{This number includes 1 for each clause.}. After
simplification the instance is reduced to 281,131 literals.

First, in Figure \ref{fig:num-instances} we see the number of
instances of a given size are processed by Solve Activities. There
are a total of 29,977 generated instances containing 1,290,208,238
literals. In terms of memory it means around 5GiB (4 bytes per
literal) of data only for transferring formulas. This number is
duplicated because activities build a watch list (storing which
clauses contain a given literals). Moreover, HUR (see Section
\ref{ssec:hbr}), requires an implication graph which also
duplicates the binaries.

\begin{figure}
  \centering
  \includegraphics[angle=-90,width=\textwidth]{data/other/instances}
  \caption{Number of instances processed versus instance size.}
  \label{fig:num-instances}
\end{figure}

Next, in Figure \ref{fig:num-depth} we see at which depth of the
search tree instances are processed by Solve activities and how many
of them are found to be inconsistent (depth is the number variables
are in the partial assignment). We see that most instances are
clustered in the range 10 - 50 with a long right tail up to 120.


\begin{figure}
  \centering
  \includegraphics[angle=-90,width=\textwidth]{data/other/depth}
  \caption{Number of instances and number of conflicts versus depth.}
  \label{fig:num-depth}
\end{figure}


Last figure \ref{fig:num-learned} shows the histogram of number
learned clauses over clause size. We see that STRUCTure learns lots
of small clauses. This happens because at lower depths STRUCTure
also learns clauses resulted from the reasoning procedures, while
at higher depths it learns only from conflicts. Nonetheless, many of
the small learned clauses are redundant. 17,406 clauses were learned
from 21,256 conflicts \footnote{Merging conflicts (see Section
\ref{ssec:learning}) helps keeping the number of learned clauses
low.}. Overall, the number of clauses learned from conflicts is much
smaller than in CD/CL solvers \cite{Marques-silva99grasp:a}.


\begin{figure}
  \centering
  \includegraphics[angle=-90,width=\textwidth]{data/other/learned}
  \caption{Histogram of number of learned clauses of clause length.}
  \label{fig:num-learned}
\end{figure}



\section{Parallel and Distributed Performance}

In this section STRUCTure is tested in a parallel and distributed
environment. To conserve resources especially with higher number
of nodes we used a set of 266 instances solved by STRUCTure on
previous tests including tests with large time limit. These instances
are enough to understand scalability of STRUCTure.

Three types of tests were conducted:
\begin{itemize}
  \item 1 node and varying number of cores used (Figure \ref{fig:para-1X});
  \item varying number of nodes and 1 core per node used (Figure \ref{fig:para-X1}); and
  \item varying number of nodes and 8 cores per node used (Figure \ref{fig:para-X8}).
\end{itemize}
These tests are meant to stress STRUCTure in a parallel, a distributed
and, respectively, a hybrid environment.

First, doubling the number of cores used on one node (Figure
\ref{fig:para-1X}) adds about 25 instances solved under 900s.
Similarly, doubling the number of nodes and using one core per node
(Figure \ref{fig:para-X1}) adds about 20 instances solved under 900s.

\begin{figure}
  \centering
  \plot{para-1X}
  \caption{Scalability in number of cores per node with 1 node used.}
  \label{fig:para-1X}
\end{figure}

The scalability is worse when the number of nodes is increased
(see Figure \ref{fig:para-X1}) compared to when the number of
cores per node is increased. One possible explanation is that in
the former case search is more in breath, while adding more cores
per node makes search going more in depth sometimes leading to a
faster solution or shorter learned clauses.

\begin{figure}
  \centering
  \plot{para-X1}
  \caption{Scalability in number of nodes with 1 core per node used.}
  \label{fig:para-X1}
\end{figure}

In Figure \ref{fig:para-X8} we see that STRUCTure maintains
its scalability in a hybrid parallel distributed environment.
Doubling the number of nodes and using eight cores per node adds
about 30 instances solved under 900s.

\begin{figure}
  \centering
  \plot{para-X8}
  \caption{Scalability in number of nodes with 8 cores per node used.}
  \label{fig:para-X8}
\end{figure}

From all figures we see that scalability is much better for random
instances than for nonrandom instances. This is because STRUCTure
parallelizes the search, but nonrandom instances benefit greatly
from the simplification procedures which are sequential (see section
\ref{sec:perf-reasoning}). On the other hand, random instances are
solved entirely by the searching component.


\section{Large Time Limit}

\emph{Large Time Limit} tests have two purposes:
\begin{enumerate}
  \item understand performance beyond default 15 minute time limit; and
  \item understand performance of using all 16 logical cores versus
  using only 8 of them.
\end{enumerate}

In Figure \ref{fig:large} the graphs of solved instances over time is
plotted for all, random and nonrandom instances when 8, 16 and 24
threads are used with a time limit of 45 minutes.

In the graphs an improvement is observed when using all 16 logical
cores versus only 8 of them. The improvement comes mostly from
random instances while nonrandom instances are barely affected
by the extra threads.  Additionally, the increase in the number
of random instances solved with 24 threads suggests that random
instances benefit from searching more in breadth and not from extra
computational resources.

Using 16 threads STRUCTure solves 187 instances in 15 minutes and 63
more in 45 minutes. 49 out of 63 instances solved with the addition
30 minutes where from the random category, which is much more than
expected if both categories benefited equally from the extra time.

\begin{figure}
  \centering
  \plot{large}
  \caption{Performance with large time limit with 8, 16 and 24
  threads on a machine with 8 physical and 16 logical.}
  \label{fig:large}
\end{figure}


\section{Comparison with Other SAT Solver}

In this section will compare STRUCTure's best performance against
two other solvers:
\begin{itemize}
  \item \emph{Cryptominisat 2.9.0} \cite{mine:cryptominisat} is a
  CD/CL SAT solver that won one gold and one silver medal medal at
  SAT Competition 2011 (\url{http://www.satcompetition.org/2011/}).

  \item \emph{March\_hi} \cite{mine:march-hi} is a look-ahead SAT
  solver which also won several medals in the past editions of SAT
  Competition random and crafted category.
\end{itemize}

All solvers are evaluated using a single node from DAS-4 cluster.
STRUCTure uses 16 threads, while Cryptominisat and March\_hi
use only 1 thread. Every instance was timed once.

STRUCTure is more similar to March\_hi in the sense that it relies on
extensive reasoning before branching. Both STRUCTure and March\_hi
perform better on random instances, while Cryptominisat performs
better on industrial instances.  Compared to Cryptominisat, STRUCTure
performs better on random instances and much worse on nonrandom
instances. Compared to March\_hi, STRUCTure performs worse on random
instances, but slightly better on nonrandom instances. Even with
extra processors STRUCTure does not perform better on champions'
categories.

Overall STRUCTure using 16 threads on 8 cores does not perform
as well as sequential champions. The sequential performance (see
Figure \ref{fig:para-1X}) is worse, but with additional cores the
performance gap closes fast.


\begin{figure}
  \centering
  \plot{compare}
  \caption{STRUCTure (best options, 8 cores) versus Cryptominisat
  2.9.0 and March\_hi (1 core)}
  \label{fig:compare}
\end{figure}
