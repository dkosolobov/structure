\chapter{Experimental Results}
\label{chap:results}

\newcommand{\plot}[1]{
  \subfigure{
    \includegraphics[width=0.5\textwidth,angle=-90]{data/all/#1}
  }

  \subfigure{
    \includegraphics[width=0.5\textwidth,angle=-90]{data/random/#1}
  }

  \subfigure{
    \includegraphics[width=0.5\textwidth,angle=-90]{data/nonrandom/#1}
  }
}

\begin{enumerate}
  \item DONE 1 node, 1..8 threads
  \item DONE 1..8 nodes, 1 thread
  \item DONE 1 node, 24 threads, 45 minutes
  \item DONE 1 node, 16 threads, nohur, nosss
  \item DONE 1..8 nodes, 8 threads
  \item DONE ve-bce
  \item TODO comparison with cryptominisat and manysat
\end{enumerate}

\section{Setup}

STRUCTure was evaluated on DAS-4 (\url{http://www.cs.vu.nl/das4/})
cluster at VU University Amsterdam (\url{http://www.few.vu.nl})
which is composed of 74 machines with 24GiB of memory and with
dual quad core Intel E5620 processors running at 2.4GHz. The
cluster allows evaluation of different types of parallelization:
\begin{inparaenum}[1)]
  \item when more nodes are added and
  \item when more CPUs per node are added.
\end{inparaenum}

The nodes have hyper threading enabled so each of them has \emph{16
logical cores}. A pair of logical cores on the same physical core
share main execution resources, but not the architectural state
(e.g. control and general purpose registers).

Except where otherwise indicated tests will run on \emph{one
node} with \emph{16 threads} (to employ all 16 logical
cores). Constellation does not spawn more than specified number
of threads. However, Java Virtual Machine will use the additional
computing power for things like garbage collector.

STRUCTure was evaluated using 678 problems
of medium difficulty from SAT Competition 2009
(\url{http://www.satcompetition.org/2009/}). The instances are
divided in three categories:
\begin{itemize}
  \item formulas in \emph{application} category encode different
  real-life problems and are typically very large (up to tens of
  millions of variables and clauses);
  \item \emph{crafted} SAT instances are generally small and designed
  to stress solvers;
  \item \emph{random} instances contain random $k$-SAT formulas.
\end{itemize}

In the following tests, instances were divided into random and
nonrandom (application and crafted) categories. There are 380 random
instances and 298 nonrandom instances.

On tests with a large time limit of 45 minutes all instances were
measured once to conserve resources. On all other tests the time
limit was 15 minutes and time measured is the average of three runs.
If for STRUCTure timed out on any run of an instance that instance
is not used for the graphs.  \footnote{The extra time give
the illusion that STRUCTure does better in 15 minutes on tests
with larger time limit than on tests with smaller time limit.
In reality because the performance varies instances which require
a time close to the upper time limit might or might not be included
in the graphs.}

The time limit in SAT Competitions is usually 20 minutes, but
the rules of DAS-4 cluster allow only 15 minutes jobs during the
daytime.  With extra 5 minutes STRUCTure can solve only few extra
instances so the additional time is not required to understand
performance a scalability.

With a time limit of 15 minutes, STRUCTure can solve shy of 190
problems on a single DAS-4 node, but state-of-art SAT solvers solve
250. With a larger time limit of 45 minutes STRUCTure solves about
250 instances.


\section{Performance of Reasonings}

\todo{HUR, SSS}

An important part of any sat solver are the simplification
procedures and reasoning it performs. Chapter \ref{chap:sat},
\nameref{chap:sat}, and subsection \ref{ssec:learning},
\nameref{ssec:learning}, describe the algorithms built in STRUCTure.

Before continuing with the test results, note
that there is a huge overlap between different techniques (e.g. BCE
implies PL and DVR \cite{Jarvisalo_blockedclause}), but they are
performed together because some are faster or perform better in
different scenarios.

In Figure \ref{fig:disable} the graph of solved instances over time
for all, random and nonrandom instances with different simplification
procedures disabled. The table \ref{tbl:disable} lists which
simplification procedures improve performance. The simplification
procedures are divided in two categories:
\begin{itemize}
  \item before search in Simplification component: DVR, BCE and VE;
  \item during search in Searching component: Split, learning, and SSS and HUR in
  Solve Activity.
\end{itemize}

The default line represents behaviour when \emph{all} reasoning
procedures are enabled. The other lines represent the behaviour
when one type of reasoning is disabled.

\begin{figure}
  \centering
  \plot{disable}
  \caption{Performance with different simplification procedures disabled}
  \label{fig:disable}
\end{figure}

\begin{table}
  \centering
  \begin{tabular}{| c | c | c | c |}
    \hline
    \multicolumn{4}{|c|}{\textbf{Simplification component (before search)}} \\
    \hline
    & \textbf{all} & \textbf{random} & \textbf{nonrandom} \\
    \hline
    Blocked Clause Elimination & - & - & = (?) \\
    Variable Elimination & + & = & + \\
    Dependent Variable Elimination & - & = & - \\
    \hline
    \hline
    \multicolumn{4}{|c|}{\textbf{Searching component (during search)}} \\
    \hline
    & \textbf{all} & \textbf{random} & \textbf{nonrandom} \\
    \hline
    Split & - & - & - \\
    Learning & ++ & ++ & ++ \\
    Self-Subsumption (in Solve activity) & - & - & + \\
    Hyper Unit Resolution (in Solve activity) & + & = & ++ \\
    \hline
  \end{tabular}

  \caption{Effect on performance of different simplification procedures.
  - bad, = no change, + good, ++ extremely good.}
  \label{tbl:disable}
\end{table}

First, from all reasoning techniques, learning has the biggest performance
impact. The gain is larger on random instances than on nonrandom
instances, though learning was expected to benefit more industrial
instances\cite{DBLP:series/faia/SilvaLM09}.

At the other extreme we notice that trying to split instances doesn't
provide any real benefit and, therefore, Split activity should be disabled
on most problem types.

VE improves performance on nonrandom instances and has almost no
impact on random instances. STRUCTure performs worse with DVR
enabled probably because the combination of BCE and VE outperforms
DVR. BCE gives worse performance on random instances, but on
nonrandom instances the performance is not affected.

Testing combination of reasoning procedures is future work.


\section{Failed Literal Probing}

We continue next to investigate the influence of Failed
Literal Probing (see subsubsection \ref{sssec:flp}) on overall
performance. In Figure \ref{fig:flp} the graph of solved instances
over time is plotted for all, random and nonrandom instances with
varying number of probes.

On random $k$-SAT instances FLP works best with high number of
probed literals. Since FLP implementation uses a weak form of BCP
it is important that instances are simplified so that they contain
at least some binaries.

On non-random instances FLP performs best with fewer probes,
i.e. only 32.  Unlike on random instances FLP works worse when
number of probes is increased.

Overall probing 128 variables gives best results, but the performance
doesn't vary much with different number of probes.

\begin{figure}
  \centering
  \plot{flp}
  \caption{Performance when varying number literal probes. (16 logical cores)}
  \label{fig:flp}
\end{figure}


\section{Large Time Limit}

\emph{Large Time Limit} tests have two purposes:
\begin{enumerate}
  \item understand performance beyond default 15 minute time limit; and
  \item understand performance of using all 16 logical cores versus
  using only 8 of them.
\end{enumerate}

In Figure \ref{fig:large} the graphs of solved instances over time is
plotted for all, random and nonrandom instances when 8, 16 and 24
threads are used with a time limit of 45 minutes.

In the graphs an improvement is observed when using all 16 logical
cores versus only 8 of them. The improvement comes mostly from
random instances while nonrandom instances are barely affected
by the extra threads.  Additionally, the increase in the number
of random instances solved with 24 threads suggests that random
instances might benefit from searching more in breadth and not from
extra computational resources.

Using 16 threads STRUCTure solves 187 instances in 15 minutes and 63
more in 45 minutes. 49 out of 63 instances solved with the addition
30 minutes where from the random category, which is much more than
expected if both categories benefited equally from the extra time.

\begin{figure}
  \centering
  \plot{large}
  \caption{Performance with large time limit with 8, 16 and 24
  threads on a 16 logical cores node.}
  \label{fig:large}
\end{figure}


\section{Parallel and Distributed Performance}

In this section STRUCTure is tested in a parallel and distributed
environment. To conserve resources especially with higher number
of nodes we used a set of 266 instances solved by STRUCTure on
previous tests including tests with large time limit. These instances
are enough to understand scalability of STRUCTure.

Three types of tests were conducted:
\begin{itemize}
  \item 1 node and varying number of cores used (Figure \ref{fig:para-1X});
  \item varying number of nodes and 1 core per node used (Figure \ref{fig:para-X1}); and
  \item varying number of nodes and 8 cores per node used (Figure \ref{fig:para-X8}).
\end{itemize}
These tests are meant to stress STRUCTure in a parallel, a distributed
and, respectively, a hybrid environment.

\begin{figure}
  \centering
  \plot{para-1X}
  \caption{Scalability in number of cores per node with 1 node used.}
  \label{fig:para-1X}
\end{figure}

\begin{figure}
  \centering
  \plot{para-X1}
  \caption{Scalability in number of nodes with 1 core per node used.}
  \label{fig:para-X1}
\end{figure}

\begin{figure}
  \centering
  \plot{para-X8}
  \caption{Scalability in number of nodes with 8 core per node used.}
  \label{fig:para-X8}
\end{figure}
  



\section{Other}

Distribution of size of learned clauses 

Distribution of depth

Distribution of instances size
