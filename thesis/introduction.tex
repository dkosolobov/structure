\chapter{Introduction}

\section{Background and Motivation}

The Boolean satisfiability (SAT) problem asks for a satisfying
assignment for a given Boolean formula or determine
that no such assignment exists and the formula is inconsistent.

SAT has applications in Computer Aided Design,
planning, routing, software testing, synthesis, theorem
proving or computational biology \cite{Smith_diagnosis,
Soeken:2010:VUM:1870926.1871248, demoura2008z3, Corblin07asat-based,
Kautz:1992:PS:145448.146725}.

For example, an interesting problem is coloring vertices
of an undirected graph with at most $k$ colors such that no
two adjacent vertices have the same color (e.g. see Figure
\ref{fig:color-graph}). For example coloring with three colors is
expressed as a SAT problem as follows:
\begin{itemize}
  \item For each vertex assign three variables (one for each color): $u, v, t$
  \item Each vertex has at least three color: $(u \lor v \lor t)$.
  \item Each vertex cannot use two colors:
  $(\neg u \lor v) \land (\neg u \lor t)$ etc.
  \item Two adjacent vertices, $A$ and $B$, must have different colors:
  $(\neg u_A \lor \neg u_B) \land (\neg v_B \lor \neg v_B) \land (\neg t_B \lor \neg t_B)$.
\end{itemize}

\begin{figure}
  \centering
  \includegraphics[width=0.3\linewidth]{dia/color-graph.eps}
  \caption{Graph Coloring problem can be reduced to SAT}
  \label{fig:color-graph}
\end{figure}

A Boolean formula is in conjunctive normal form (or
CNF) if it is a conjunction of clauses. A clause is a disjunction
of literals. A literal is a variable, or the negation of a
variable. $k$-SAT is the problem of finding a satisfying assignment
for a CNF formula where all clauses have at most $k$ literals.

SAT is the first and one of the simplest NP-complete problem
\cite{Cook:1971:CTP:800157.805047}. It is known that $2$-SAT is in
P \cite{karp}, while for $k \ge 3$ only exponential time algorithms
are known, such as Sch\"{o}ning algorithm which runs in $O((2\frac{k
- 1}{k})^n \cdot poly(n))$ where $n$ is the number of variables
\cite{Schoning:1999:PAK:795665.796524}.

Many problems, such as "independent set problem", are proven to be
NP-complete by encoding SAT into that problem in polynomial time.
There have been many failed attempts to find polynomial time
algorithms for $k$-SAT in order to prove that P=NP.  \footnote{The
following site lists many attempts at providing a polynomial solution
to SAT. \url{http://www.win.tue.nl/~gwoegi/P-versus-NP.htm}}

While industrial problems can be quite large (on order
of tens of thousands of variables) some classes, such as
\emph{Automatic Test Pattern Generation}, are very easy to solve
in practice (with a running time $O(n^3)$) because they exhibit
hidden structure easily exploitable by current SAT solvers
\cite{Prasad:1999:WAE:309847.309857}.

While state of the art sequential solvers are fast there have
been no recent algorithmic improvements which had provided major
speed-ups gains. On the other hand new applications challenge modern
SAT solvers with more complex problems.

The lack of performance increase is accentuated by the recent
architecture shift from uni-core to multi-cores systems. Few
years ago we hit the thermal wall: increasing frequency leads to
higher power consumption and more heat generated. While Moore's law
continued, processor manufactures used the extra transistors to build
multi-core chips while individual cores may become slower.
Therefore, SAT solvers need to adapt to this new paradigm.

\emph{Scaling-out} is the ability to do tasks faster when
more nodes or CPUs are added to the distributed process
\cite{citeulike:1567858}. There are two main ways to scale
out:
\begin{inparaenum}[a)]
  \item increase number of CPUs and/or cores per node; or
  \item adding multiple nodes to the distributed process.
\end{inparaenum} Each of these have different implications
for the development of a parallel SAT solver.

STRUCTure is a distributed SAT solver built at VU University
Amsterdam.  STRUCTure aims at parallelizing the search and not the
reasoning algorithms. This simplifies some reasoning procedures
such as Pure Literals Rule (see \ref{ssec:pl}) which are inherently
sequential \cite{Johannsen:2005:CPL:1166822.1166834}.

STRUCTure is built on top of Constellation, a distributed
scalable message passing distributed programming framework.


\section{Contributions}

\begin{itemize}
  \item distributed searching
  \item distributed learning
  \item hbr/hur algorithm
\end{itemize}

\section{Overview}

Chapter \ref{chap:related}, \nameref{chap:related}, provides an
overview of the state-of-art techniques used in modern SAT solvers.

Chapter \ref{chap:sat}, \nameref{chap:sat}, formalizes the SAT
problem and explains common simplification and reasonings techniques.

Next, chapter \nameref{chap:architecture} describes the architecture
of STRUCTure in great detail. Constellation, a scalable distributed
programming framework, is also presented here.

In chapter \ref{chap:results}, \nameref{chap:results}, STRUCTure is
run with SAT instances from SAT Competition 2009. Simplification
procedures and reasoning techniques implemented in STRUCTure are
evaluated. Moreover, the scalability of STRUCTure is tested
using different combinations of number of nodes and number of cores
per node.

The following chapter, \ref{chap:future}, lists some ideas that can
improve STRUCTure's performance and make it competitive with respect
to the state of art solvers. Finally, in chapter \ref{chap:conclusions}
some conclusions are drawn.
