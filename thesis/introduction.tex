\chapter{Introduction}

\section{Background and Motivation}

The Boolean satisfiability problem asks for a satisfying
assignment for a given Boolean formula or determine
that no such assignment exists and the formula is inconsistent.

A Boolean formula is in conjunctive normal form (or
CNF) if it is a conjunction of clauses. A clause is a disjunction
of literals. A literal is a variable, or the negation of a
variable. $k$-SAT is the problem of finding a satisfying assignment
for a CNF formula where all clauses have at most $k$ literals.

For example, an interesting problem is coloring vertices
of an undirected graph with at most $k$ colors such that no
two adjacent vertices have the same color (e.g. see figure
\ref{fig:color-graph}). This problem is easily encoded
in $k$-SAT. For example coloring with three colors:
\begin{itemize}
  \item To each vertex assign three variables (one for each color): $u, v, t$
  \item Each vertex has at least three color: $(u \lor v \lor t)$.
  \item Each vertex cannot use two colors:
  $(\neg u \lor v) \land (\neg u \lor t)$ etc.
  \item Two adjacent vertices, $A$ and $B$, must have different colors:
  $(\neg u_A \lor \neg u_B) \land (\neg v_B \lor \neg v_B) \land (\neg t_B \lor \neg t_B)$.
\end{itemize}

\begin{figure}
  \centering
  \includegraphics[width=0.3\linewidth]{dia/color-graph.eps}
  \caption{Graph Coloring problem can be reduced to SAT}
  \label{fig:color-graph}
\end{figure}

SAT is the first and one of the simplest NP-complete problem
\cite{Cook:1971:CTP:800157.805047}. It is known that
$2$-SAT is in P \cite{karp}, while for $k \ge 3$ the best
known algorithm, due to Sch\"{o}ning, runs in $O((2\frac{k -
1}{k})^n \cdot poly(n))$ where $n$ is the number of variables
\cite{Schoning:1999:PAK:795665.796524}.
 
Many problems, such as "independent set problem", are proven to be
NP-complete by encoding SAT into that problem in polynomial time.
There have been many failed attempts to find polynomial time
algorithms for $k$-SAT to prove that P=NP.  \footnote{The following
site lists many attempts at providing a polynomial solution to
SAT. \url{http://www.win.tue.nl/~gwoegi/P-versus-NP.htm}}

SAT has applications in Computer Aided Design, planning, routing,
software testing, synthesis, theorem proving or computational
biology \cite{Smith_diagnosis, Soeken:2010:VUM:1870926.1871248,
demoura2008z3, Corblin07asat-based}. While industrial problems
can be quite large (on order of tens of thousands of variables)
some classes, such as \emph{Automatic Test Pattern Generation},
are very easy to solve in practice (with a running time $O(n^3)$)
because they exhibit hidden structure easily exploitable by current
SAT solvers \cite{Prasad:1999:WAE:309847.309857}.



\section{Case for parallelism}

State of the art sequential solvers are fast, but there are no
recent algorithmic improvements which have provided major speed-ups
gains. On the other hand new applications challenge modern SAT
solvers with more complex problems.

The lack of performance increase is accentuated by the recent
architecture shift from uni-core to multi-cores systems. Few
years ago we hit the thermal wall: increasing frequency leads to
higher power consumption and more heat generated. While Moore's law
continued, processor manufactures used the extra transistors to build
multi-core chips.  SAT solvers need to adapt to this new paradigm.

\emph{Horizontal scalability} is the ability to do tasks faster
when more nodes or CPUs are added to the distributed process. There
are two main ways to scale horizontally:
\begin{inparaenum}[a)]
  \item increase number of CPUs and/or cores per node; or
  \item adding multiple nodes to the distributed process.
\end{inparaenum} Each of these have different implications
for the development of a parallel SAT solver.

STRUCTure aims at parallelizing the search and not the reasoning
algorithms. Indeed some reasonings such as Pure Literal Rule (see
\ref{ssec:pl}) are inherently sequential and cannot be parallelized
in the general case \cite{Johannsen:2005:CPL:1166822.1166834}. Even
if some parallelization can be obtained, the gain would be marginal
because STRUCTure has only few sequential points.

In order to scale we need to avoid distributed shared memory,
broadcasting and synchronization. Constellation, the distributed
framework STRUCTure is built on, provides a scalable message passing
distributed programming model.

\section{Contributions}

\section{Overview}
