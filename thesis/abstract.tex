\begin{abstract}

The problem of Boolean satisfiability (SAT), though NP-complete,
has many real-life applications in planning, designing, synthesis
and routing. Due to the widespread success of the SAT problem
it is important to solve it very fast even on large and complex
applications.

In this thesis we introduce STRUCTure, a new parallel and distributed
SAT solver built to take advantage of the multi-cores processors
available in today's computers. In STRUCTure the divide-and-conquer
paradigm is used to parallelize the search. The solver is built
on top of Constellation, a distributed computing framework based
on message passing. Constellation handles load balancing and work
distribution which makes scalability easy.

STRUCTure performs many reasoning and simplification procedures
before and during the search. Branch selection heuristics
are vital for a small searching space. STRUCTure has a unique
built-in distributed clause learning system required to solve some
problems. To remove the need for synchronization, clauses are learned
while searching, but incorporated after the search is restarted.

Our solver has been tested on the DAS-4 cluster at VU University
Amsterdam. Good scalability has been observed in
\begin{inparaenum}[a)]
  \item a parallel environment (1 to 8 cores on one node);
  \item a distributed environment (1 to 8 nodes using 1 core each); and
  \item a hybrid environment (1 to 8 nodes using 8 cores each).
\end{inparaenum}
STRUCTure performs slightly better in a parallel environment than in
a distributed environment.  Similar to look-ahead solvers, our solver
performs and scales better on random instances than on industrial
applications. Compared to awards winning solvers STRUCTure performs
worse overall, but there is room for improvement considering that
STRUCTure is still in its infancy.

We show that a scalable SAT solver can be built based on the
divide-and-conquer paradigm using a framework designed to parallelize
such algorithms.

\end{abstract}
