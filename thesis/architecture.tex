\chapter{Architecture}

A top view of STRUCTure's architecture is shown in figure
\ref{fig:architecture}. There are main four components
out of which only \emph{Search Backtrack} is essential.
The other three are optimization to handle larger
industrial and crafted instances.
\begin{enumerate}
  \item \emph{Circuit Preprocessing} reverts and simplifies 
  part of the original Boolean circuit;
  \item \emph{Simplification} performs formula simplifications;
  \item \emph{Restart} starts and stops search and extends
  formula with clauses derived from conflicts;
  \item \emph{Search Backtrack} performs a distributed solution search.
\end{enumerate}

\begin{figure}
  \centering
  \includegraphics[width=\textwidth]{dia/architecture}
  \caption{STRUCTure has four components. \emph{Searching}
  is mandatory, the other three are optimizations
  for industrial class instances.}
  \label{fig:architecture}
\end{figure}


\section{Constellation}

Constellation is a distributed programming framework developed at VU University
Amsterdam \cite{mine:constellation}. In Constellation the unit of work is
an activity. Activities can perform some work, spawn new activities,
send messages to other activities and process incoming messages
(see figure \ref{fig:constellation}).

Constellation doesn't provide any shared memory mechanism, but
activities running on the same node are free to use the common
address space. Moreover, broadcasting or cancellation is not
available, but are needed in STRUCTure (see \ref{sssec:blackhole})
to stop the search.

Constellation works by running a fixed number of worker threads
on each node which execute activities.  Workers maintain a
queue of events (such as new activities or messages passed) which are
processed in last-in-first-out (LIFO) order.

When a thread runs out of work it attempts to steal more
jobs first from other workers on the same node, and then from
other workers on different nodes.  Local steals are performed
in LIFO order, while remote steals are performed in FIFO
(first-in-first-out) order. Remote steals are
slower than local steals because they involve network communication
and they are performed only at regular intervals.

\begin{figure}
  \centering
  \includegraphics[width=0.4\linewidth]{dia/constellation}
  \caption{In Constellation activities spawn new
  activities and communicate by message passing.
  Activities can be executed on any node (represented by
  distinct color) participating in the distributed process.}
  \label{fig:constellation}
\end{figure}


\section{Activities}

In STRUCTure an activity's goal is to find a satisfying assignment
for an instance, prove unsatisfiability or learn something new
about the instance. Some activities can spawn new activities to
solve smaller instances.  Learning is achieved when climbing back
on the searching tree.

Activities return to parent a solution to instances which can be
\textsf{SATISFIABLE}, \textsf{UNSATISFIABLE} or \textsf{UNKNOWN}. If
a satisfying assignment was found the solution includes units,
otherwise it includes learned clauses from conflicts.


\begin{figure}
  \centering
  \includegraphics[width=0.9\linewidth]{dia/activity}
  \caption{Activities in STRUCTure. Rectangles represent activities, arrows
  show dependence among activities.}
  \label{fig:activities}
\end{figure}


\subsection{Circuit Preprocessing}

Many industrial applications encode logic circuits as CNF formulas.
All gates except XOR gates require a number of clauses linear in
the number of inputs and are handled easily by most solvers. On
the other hand, XOR gates require an exponential number of clauses
and many solvers have difficulty handling them.

This part of the solver attempts to reverse and simplify part of the
original circuit using XOR gates extraction and dependent variable
removal described in \ref{ssec:xor-extraction} and \ref{ssec:dvr}.

Unfortunately, many reasoning algorithms do not handle XOR gates
well so the after DVR the remaining gates are turned back into CNF
clauses. There is room for improvement because DVR will remove
variables introduced to split large XOR gates so formula might get
much larger.


\subsection{Restart Loop}

Restart Loop does three things:
\begin{inparaenum}[a)]
  \item simplifies the formula;
  \item starts and stops distributed search; and
  \item adds learned clauses and repeats the process with the enlarged formula.
\end{inparaenum} Each iteration of the loop is called a
\emph{generation}. The enlarged formula is passed through the
simplification steps to strengthen the learned clauses and simplify
formula with the new information.

\emph{Blocked clauses elimination} activity (BCE) performs
blocked clause elimination to remove redundant clauses (see
\ref{ssec:bce}). \emph{Variable elimination} activity (VE) performs
variable elimination to remove redundant variable (see
\ref{ssec:ve}). \emph{Simplify} activity is similar to Solve
activity from the searching backtrack (see \ref{sssec:solve}).
Subsection \ref{sssec:blackhole}, BlackHole activity, describes
How the distributed computation is stopped.

Currently the time to live of a generation is fixed to 10 second
plus 5 seconds for each new generation. \todo{Make tests to improve
restart} The restart loop is sequential, but because generation
periods increase and the first iteration is the most expensive the
fraction of time spent in the loop decreases with larger problems.


\subsection{Look-Ahead}

\emph{Look-Ahead}'s goal is to learn new units and binaries.
It performs a small look ahead at the root of the searching tree. It
picks several variables (around 128 in current implementation)
and propagates them using the \emph{Propagate} activity.

Given a variable $u$ Propagate activity performs boolean constraint
propagations on $u$ and $\neg u$. Variables to be tested are chosen
by Look-Ahead activity using the same variable selection heuristic
used for branching (see \ref{ssec:branching}).

If $u, u_0, u_1, \ldots$ are propagated literals and $v, v_0, v_1,
\ldots \in UP(F, u)$ then Propagate activity learns the clauses in table
\ref{tbl:propagate-learn}. There are plenty of other binary clauses
to be learned (one for each literal $\in UP(F, u)$), but many of them
are redundant and don't improve performance if added to formula.

\begin{table}
  \centering
  \framebox{
    \begin{tabular}{c c c}
      initial & propagated & learned \\
      \hline
      \hline
      $u$ & $\emptyset$ & $\neg u$ \\
      \hline
      $u$ & $v$ & $v$ \\
      $\neg u$ & $v$ & \\
      \hline
      $u$ & $v$ & $u \leftrightarrow v$ \\
      $\neg u$ & $\neg v$ & \\
      \hline
      $u$ & $u_0$ & $u \rightarrow u_0$ \\
      $u_0$ & &
    \end{tabular}
  }
  \caption{Learned units and binaries by Propagate activity}
  \label{tbl:propagate-learn}
\end{table}



\subsection{Searching Backtrack}
\label{ssec:searching-tree}

The \emph{Searching backtrack} component searches for a solution.
When stopped it found a solution or learned a bunch of new clauses.
At the top of search tree there are not enough activities to occupy
all participating workers, but as more activities are spawned and
distributed more nodes join in.

In Constellation local steals are performed in LIFO order which means
that smaller jobs (more variables assigned) are stolen first. On
the other hand remote steals are performed in FIFO order corresponding
to larger jobs (less variables assigned) are stolen first.

The steal type's effect on the search is interesting. Adding
more nodes makes the search go more in breadth, while adding more
CPUs per node makes search go more in depth. On some satisfiable
instances this means that more nodes will close in on more solutions,
while more CPUs per node will close in faster on fewer solutions.
Inconsistent formulas should not be affected.

\subsubsection{Split}

The formula $F$ in the next example is a conjunction of two
independent formulas $F_1$ and $F_2$. $F$ is consistent only if both
$F_1$ and $F_2$ are consistent. If $F_1$ or $F_2$ are inconsistent
then $F$ is inconsistent.

\begin{align}
  F &= (1 \lor 2 \lor 3) \land (1 \lor \neg 4) \land (5 \lor \neg 6) \land (6 \lor \neg 7) \\
  F_1 &= (1 \lor 2 \lor 3) \land (1 \lor \neg 4) \\
  F_2 &= (5 \lor \neg 6) \land (6 \lor \neg 7) \\
  F &= F_1 \land F_2
\end{align}

\begin{myprop}
  If $F(u_1, \ldots, u_{k_1}, \ldots, u_{k_2}, \ldots, u_{k_n}) =
  F_1(u_1, \ldots, u_{k_1}) \land \ldots \land F_n(u_{k_{n-1}+1},
  \ldots, u_{k_n})$ for $1 < k_1 < \ldots <k_n$ then $F$
  is consistent whenever all instances $F_1, \ldots, F_n$
  are consistent.
\end{myprop}

Split activity implements the algorithm \ref{alg:split} and provides
an opportunity for parallelization.


\subsubsection{Select}
\label{ssec:branching}

The goal of \emph{Split} activity is to choose a branching variable
which will be assigned with both $True$ and $False$. It is preferable
that both polarities reduce the formula as much as possible after
propagation.  If for example, $\{ u \}$ is an unit clause then
choosing $u$ for branching is pointless because branch $\neg u$
will return a contradiction, while branch $u$ will not reduce the
formula further.

March \cite{mine:march} suggests doing one forward backtracking
step on a fixed number of variables (i.e. look-ahead) in order to:
\begin{inparaenum}[a)]
  \item learn new units and binaries and
  \item estimate size of formula after branching.
\end{inparaenum} The branching variable is picked such that
the formula is simplified as much as possible.

In STRUCTure performing look-ahead at every branch gives worse
performance. My intuition is that look-ahead is very expensive
compared to simply assigning a score based on clauses in which a
variable is present. Moreover, the additional reasoning in STRUCTure
before branching might outperform benefits of learning new units.

To determine the best variable for branching a measure $H_u$
is computed for every variable.  $H_u$ was inspired by a similar
measure in \cite{mine:oksolver}. The variable with the highest
score is selected for branching.

$H_u$ is computed based on measures of both variable's polarities
$S_u, S_{\neg u}$. $T_u$ is how much a formula is reduced if literal
$u$ is assigned without looking at propagations ($u \rightarrow
v$). $S_u$ improves on $T_u$ by including scores of implied units.

\begin{align}
  T_u &= \sum_{u \in C \in F}{2^{-|C|}} + \sum_{u \in C \in F, |C| = 2}{2^{-4}} \\
  S_u &= T_u + \sum_{\{ u \rightarrow v \} \in F}{T_v}
\end{align}

$T_u$ favors literals that appear in many short clauses.
$S_u$ favors literals for which implications appear in many short clauses.
Finally $H_u$ is computed using the following function:

\begin{align}
  H_u &= G(S_u, S_{\neg u}) \\
  G(p, n) &= \sum_{i=1}^{n}{(F_i(p, n) + F_i(n, p))} \\
  F_i(p, n) &= a_i \cdot p^{b_i} \cdot n^{c_i}, a_i, b_i, c_i \in \mathbb{Z} \\
  G, F_i &: \mathbb{R} \times \mathbb{R} \rightarrow \mathbb{R}, i = 1 \ldots n
\end{align}

$G$ is a generalization of the formula used in other
look-ahead solvers such as \cite{mine:march}: $G'(p, n) =
1024 \times p \cdot n + p + n$. A couple of properties $G$ must have are:
\begin{inparaenum}[a)]
  \item it must be easy (and fast) to compute (e.g. no exponentiation or logarithm); and
  \item it must be symmetrical.
\end{inparaenum}


For $n = 3$ and $a_i, b_i, c_i \in \mathbb{Z}_5$ best results were
obtained for polynomial:
\begin{align}
  G(p, n) &=  p^2 \cdot n^2 \cdot (2 \cdot p^2 \cdot n^2 + 2 \cdot p \cdot n + p + n)
\end{align}
Increasing ranges for $n, a_i, b_i, c_i$ did not result in
significant better performance.


\subsubsection{Branch}

\emph{Branch} activity uses the branching variable, $b$, to generate
two instances $F_+ = F \cup \{b\}$ and $F_- \cup \{\neg b\}$ and attempts
to solve both of them.

\begin{myprop}
  If $F$ is a CNF formula and $u$ is a branching variable
  $F$ is consistent if and only if $F_+$ or $F_-$ are consistent.
\end{myprop}

If a solution is found branch activity forwards it to parent before
remaining branch (if any) is completed. Constellation cannot cancel
activities, but local executors perform events in LIFO order so
the solution is propagated fast to Restart activity which will stop
the computation.


\subsubsection{BlackHole}
\label{sssec:blackhole}

\emph{BlackHole} activity kills instances at the end of one
generation by immediately returning \textsf{UNKNOWN} instead of to
solving them.

\emph{BlackHole} hits the limits of Constellation distributed
model because in order to stop current computation all instances of
BlackHole activity on all nodes must know that the computation has
ended, but activities know little about nodes executing them. In
a shared memory model this is a matter of flipping a flag, but with
different address spaces the Restart activity must broadcast its
intent to stop the distributed computation.

In STRUCTure broadcast is implemented like in figure
\ref{fig:broadcast}.  All instances are accompanied by an
identifier of an unique \emph{TracerMaster} activity responsible for
broadcasting stop notifications. On every node a \emph{TracerSlave}
activity is responsible for receiving the notifications. When a
node executes its first activity TracerSlave registers itself to
TracerMaster which maintains a list of all available slaves. At the
end of current generation Register activity contacts TracerMaster
which contacts all registered Slaves.

\begin{figure}[h]
  \centering
  \includegraphics[width=0.5\linewidth]{dia/broadcast}
  \caption{Broadcast is implemented with
  TracerSlave on each node and a single TracerMaster}
  \label{fig:broadcast}
\end{figure}


\subsubsection{Solve}
\label{sssec:solve}

\emph{Solve} activity attempts to find a satisfying assignment for
received formula. If a solution (i.e.  formula is determined to be
consistent or inconsistent) is found it is propagated back up on
the search tree. If the formula is inconsistent it is called that
a \emph{conflict} was reached with current variable assignments.

If the satisfiability of formula cannot be determined the activity
generates a smaller formula, the \emph{core}. The solution of the
core is used to reconstruct a solution of the original
formula.

In the original backtracking algorithm in
\cite{Davis:1960:CPQ:321033.321034} this activity corresponds to
the Boolean constraint propagation step, but STRUCTure performs
additional simplification steps:
\begin{itemize}
  \item Boolean constraint propagation (\ref{ssec:bcp});
  \item Hyper Unit Resolution (\ref{ssec:hbr});
  \item Equivalent Literal Renaming (\ref{ssec:eqlr});
  \item Hidden Tautology Elimination (\ref{ssec:hte});
  \item Subsumming and Self-Subsumming (\ref{ssec:sss});
  \item Pure Literal Rule (\ref{ssec:pl}).
\end{itemize}


\subsubsection{Learning}
\label{sssec:learning}

When the searching backtrack encounter a conflict the current
variable assignment cannot lead to a satisfying assignment.
Take for example the searching tree in figure \ref{fig:learning}.
Assignment of literals $1, 2, \neg 4$ lead to a conflict so at
least one of the variables must have a different value and hence the
clause $\neg 1 \lor \neg 2 \lor 4$.  In a future generation when 2
out of the 3 variables involved in the conflict are assigned the same
value the last variable is constrained to have a different value.

\begin{figure}
  \centering
  \includegraphics[width=0.5\linewidth]{dia/learning}
  \caption{Conflicts are transformed into clauses by forbidding
  same variable assignment.
  $\overline{(1 \land 2 \land \neg 4)} \equiv (\neg 1 \lor \neg 2 \lor 4)$}
  \label{fig:learning}
\end{figure}

Now consider two conflicts $C_1 = \{ \neg 1, \neg 2, 4 \}$ and
$C_2 = \{ \neg 1, 2 \}$.  By applying resolution (see definition
\ref{mydef:resolution}) clause $C_3 = \{ \neg 1, 4 \}$ is obtained
which is a subset of $C_1$; therefore $C_1$ can be replaced by $C_3$
strengthening the learned clause. Similarly conflicts
$C_4 = \{ 1, 3, 5 \}$ and $C_5 = \{1, 3, \neg 5\}$ are merged
into $C_6 = \{ 1, 3 \}$.

\begin{myprop}
  If assignment of literals $u_1, \ldots u_i$, $i \ge 0$, leads to
  a conflict then the \emph{conflict clause} $\{ \neg u_1, \ldots
  \neg u_i \}$ must be satisfied by any satisfiable assignment of
  the original formula.
\end{myprop}

\begin{myprop}
  If $C_1 = \{ v_1, \ldots, v_i, \mathbf{u} \}$ and $C_2 = \{
  v_1, \ldots, v_i, \mathbf{\neg u}, v_{i+1}, \ldots v_{i+j} \}$
  are two conflict clauses, $i, j \ge 0$, then $C = \{ v_1, v_2,
  \ldots, v_{i+j} \}$ must also be satisfied.
\end{myprop}

