\chapter{Architecture}

\section{Constellation}

Constellation's parallel programming model is based on activity spawning.
Communication is done through message passing among running activities. There
is no shared memory available, though activities running on the same machine
could make use of the common address space.  For the SAT Competition 2011
STRUCTure uses a single machine with up to number of cores
specified activities running simultaneously.


\section{Activities}

An activity goal's is to find a satifying assignment for an instance,
detect unsatisfiability or learn something new about the instance.
Some activities can spawn new activities to solve smaller instances.
Learning is achieved when climbing back on the searching tree.

Activities return to parent a solution to instances which can be
\emph{SATISFIABLE}, \emph{UNSATISFIABLE} or \emph{UNKOWN}. If
a satisfying assignment was found the solution includes units,
otherwise it includes a learning tree. \todo{Define learning tree}

A graph dependecies between activities is shown in figure \ref{fig:activities}.


\begin{figure}
  \centering
  \includegraphics[width=1.5\linewidth]{activity.pdf}
  \caption{Activities in STRUCTure}
  \label{fig:activities}
\end{figure}


\subsection{Initial}

\begin{itemize}
  \item \emph{Preprocess} Performs some initial preprocessing.

\end{itemize}


\subsection{Restart loop}

\begin{itemize}
  \item \emph{XOR/DVE} extracts XOR gates and performs \emph{Dependent
  Variable Elimination}

  \item \emph{BCE} performs \emph{Blocked Clause Elimination}

  \item \emph{VE} performs \emph{Variable Elimination}

  \item \emph{Simplify} simplifies the instance.

  \item \emph{Restart} restarts searching at increasing intervals of times.
  Each instance is part of a generation number. When the timeout expires
  a death certificate for that generation is issued. When computation returns
  the learned clauses are added to current instance and the searching starts
  over with \emph{Blocked Clause Elimination}.

\end{itemize}


\subsection{Searching tree}

\begin{itemize}
  \item \emph{LookAhead} performs a small look ahead at the root
  of the searching tree. It picks a several variables (around 128 in
  current implementation) and propagates them (using the \emph{Propagate}
  activity). This activity's goal is to learn new units and binaries.

  \item \emph{Propagate} propagates a literal $l$ using
  unit propagation \todo{Describe unit propagation}. If $u$ is propagated from
  $l$ then the binary $l \rightarrow u$ is true.
  Propagate activity attemts to learn new units and binaries as follows

  \begin{itemize}
    \item If an empty clause is discovered then $\neg l$ can be implied.

    \item $(l \rightarrow u) \land (l_0 \rightarrow \neg u) \Rightarrow
    \neg l \lor \neg l_0$. If two literals $l$ and $l_0$ imply different
    assignment for the same variable the two literals cannot both be true.

    \item $(l \rightarrow u) \land (\neg l \rightarrow u) \Rightarrow u$. If
    different phases of the same variable propagate the same literal, $u$,
    then $u$ must be true.

    \item $(l \rightarrow u) \land (\neg l \rightarrow \neg u)
    \Rightarrow l \equiv u$

  \end{itemize}

  \item \emph{Split} checks if instance is composed of multiple independent
  instances. Learns the reunion of the child instances.

  \item \emph{Select} selects next variable for branching.

  \item \emph{Branch} uses the branching variable, $b$ to generate two
  instances $F_+ = F \cup \{b\}$ and $F_- \cup \{\neg b\}$. If any of
  $F_+$ or $F_-$ is satisfiable then a satisfying assignment for
  $F$ is given by the satisfying assignment of the child instance.
  If both $F_+$ and $F_-$ are unsatisfible then $F$ is unsatisfiable.
  \todo{How SelfSubSumming is perfomed}

  \item \emph{BlackHole} kills instances part of a dead generation.
  Death certificates are received from Restart activity and is important
  that different generations have different identifiers. Solution to
  a killed instance is \emph{UNKOWN}.

  \item \emph{Solve} attemts to solve the current instance. If a solution
  is found it is propagated back up on the search tree. If a solution
  is not found a smaller instance, the \emph{core}, is generated. The solution
  of the core instance can be used to deduce a solution to the main instance
  instance a solution to the instance being solved can be deduced. In case
  of contradiction Solve activity learns last branched literal.

\end{itemize}


