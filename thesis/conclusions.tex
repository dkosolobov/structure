\chapter{Conclusions}
\label{chap:conclusions}

The problem of Boolean satisfiability is the simplest NP-complete
problem and has many industrial applications such as planning,
designing, synthesis and routing. Its success is its arch enemy
as Boolean satisfiability is used on more complex and larger
applications.

STRUCTure is a Boolean satisfiability solver built with
parallelization and distribution in mind. It aims to speedup solution
search by using available cores in the modern processors. It does not
aim at parallelizing reasoning techniques since some of them are
inherently sequential. The solver is built on top of Constellation
a distributed computing framework. Constellation provides very good
scalability, but lacks a few features such as broadcasting and job
cancellation that are required by STRUCTure.

STRUCTure is based on the original DPLL algorithm. The Boolean
constraint propagation step was replaced with more advanced reasoning
techniques such as Hyper-Unit Resolution, Pure Literal Elimination
and clause Subsumption.  Additionally, STRUCTure performs other
simplification procedures such as circuit preprocessing, blocked
clause elimination, variable elimination, failed literal probing
and restarting.

To improve the performance, STRUCTure has built-in learning.  When a
conflict is reached a new clause is learned. However, the learned
clauses are added to the formula only at restart if no solution was
found. This simplifies learning in a distributed environment as no
communication is required between workers.  The extended formula is
simplified again and a new attempt to solve it is performed. Unlike
other solvers that perform clause learning, STRUCTure does not trim
learned clauses.

We evaluated STRUCTure on 678 instances from SAT Competition 2009.
The instances were divided into two categories: random and nonrandom
(includes crafted and application instances from original set).
On a single node with 8 processors STRUCTure can solve 197 instance
in under 900s.  STRUCTure performs better on random instances than
on nonrandom instances. This behaviour is consistent with look-ahead
solvers such as March.

All reasoning techniques, except learning, do not improve performance
on random instances. However, random instances have a clear benefit
from additional cores in a distributed environment which suggests
that STRUCTure is successful at parallelizing work.

On the other hand, nonrandom instances benefit from some
simplification procedures. Because the simplification procedures
are sequential scalability of STRUCTure on nonrandom instances is
worse than on random instances. It is not clear if scalability will
be maintained when simplifications and sequential case improves.

Searching in STRUCTure scales well with increasing number of nodes
and slightly better with increasing number of cores per node.
The scalability is maintained in a hybrid parallel/distributed
environment.

STRUCTure does not perform as well as awards winning solvers.
There are several reasons for this performance difference:
\begin{itemize}
  \item STRUCTure is much newer than other solvers and there are
  still many algorithmic improvements and coding tweaks to be made.
  \item Often parallelization and scalability have an additional
  overhead (e.g. work distribution, data replication, network
  communication) which decreases the sequential performance.
  \item STRUCTure runs inside a Java Virtual Machine and often
  interpreted byte code is slower than machine code (even with just
  in time compilation)
\end{itemize}


We have shown that a scalable distributed SAT solver can be built
using a message passing model. The solver needs more work to become
competitive and solve industrial strength applications.
