\chapter{Basic Concepts}
\todo{This section diserves a special chapter}

\begin{mydef}
  \emph{Boolean space} is $\mathbb{B} = \{ True, False \}$.
\end{mydef}

\begin{mydef}
  A \emph{Boolean formula} $F$ is a function of $n$ \emph{boolean variables}
  $F : \mathbb{B}^n \rightarrow \mathbb{B}$.
\end{mydef}

\begin{mydef}
  A Boolean formula $F : \mathbb{B}^n \rightarrow \mathbb{B}$ is
  \emph{consistent} (or satisfiable) if exists an assignment of
  variables such that the formula is \emph{True}
  ($\exists u_0, \ldots u_{n-1}$ such that $F(u_0, \ldots, u_{n-1}) = True$).
  F is said to be \emph{inconsistent} (or unsatisfiable) if
  there is no such assignment.
\end{mydef}

\begin{mydef}
  Two formulas $F_1$ and $F_2$ are equivalent, $F_1 \equiv F_2$, if
  any satisfying assignment of one formula is a satisfying assingment
  of the other.
\end{mydef}

\begin{mydef}
  A \emph{propositional} Boolean formula is a Boolean formula that contains only
  logic operations \textbf{and} (conjuntion, denoted with $\land$),
  \textbf{or} (disjuntion, denoted with $\lor$) and \textbf{not}
  (negation, denoted with $\neg$). The \textbf{xor} (exclusive
  disjuntion, denoted with $\oplus$) operator will also be used.
\end{mydef}

A propositional Boolean formula is in \emph{Conjunctive Normal Form}
(short CNF), if:
\begin{itemize}
  \item It is a conjuction of \emph{clauses};

  \item Each clause is a disjunction of one more \emph{literals};

  \item A literal is an occurance of a Boolean variable or its negation.
  If the literal is tthe occurance of the variable
  then it has \emph{postive polarity}, otherwise it has \emph{negative
  polarity}.

\end{itemize}

We will use small letters $u, v, t, \ldots$ or small letters
with indices $u_0, u_1, \ldots$ to indicate variables and literals
(the distinction will be obvious in the context). We will use numbers
$1, 2, \ldots$ to indicate variables in examples.

We will denote clauses as sets for literals $C = \{ u_0, u_1, \ldots \}$
and formula $F$ as a set of clauses.

An example of formula is
\[
F(u, v, t) = (u \lor v \lor t) \land (\neg u \lor \neg v)
\]
or as sets
\[
F = \{ \{u, v, t\}, \{\neg u, \neg v)\}\}
\]
\todo{Give a broader example drawing the implication graph}.

\begin{mydef}
  A clause is a \emph{tautology} if it contains a literal, $u$, and its negation
  $\neg u$. A clause of length 0 is an \emph{empty clause}; a clause of length 1
  is a \emph{unit clause}; a clause of length 2 is a \emph{binary clause}. 
\end{mydef}

\begin{myprop}
  A formula containing 0 clauses is consistent.
\end{myprop}

\begin{myprop}
  A formula containing an empty clause ($\emptyset \in F$) is inconsistent.
\end{myprop}

\begin{myprop}
  A binary clause is equivalent with two implications.
  $u \lor v \iff \neg u \rightarrow v \iff \neg v \rightarrow u$.
\end{myprop}

\begin{myprop}[Equivalent literals]
  \label{myprop:equivalent-literals}
  $(u \rightarrow v) \land (v \rightarrow u) \Rightarrow (u \leftrightarrow v)$.
\end{myprop}

\begin{myprop}[Resolution]
  $(u \lor v_1 \lor \ldots \lor v_{k_v})
  \land (\neg u \lor t_1 \lor \ldots \lor b_{k_t})
  \Rightarrow (u_1 \lor \ldots \lor u_{k_u} \lor t_1 \lor \ldots \lor t_{k_t})$
\end{myprop}

\begin{mydef}
  The \emph{implication graph} of a formula $F$ is an oriented graph
  $\I_F = (V, E)$ where $V$ is the set of all literals and $E = \{(u,
  v) \in V \times V | \{\neg u, v\} \in F\}$ (E contains all implications in $F$).
\end{mydef}

We will denote $u \rightsquigarrow v$ if there is a path from $u$ to $v$ in $\I_F$.

\begin{myprop}
  $u \rightarrow v \equiv u \rightsquigarrow v$.
\end{myprop}

\begin{myprop}[Equivalent literals in $\I_F$]
  \label{myprop:equivalent-literals-in-if}
  $u \leftrightsquigarrow v \iff u \leftrightarrow v$
\end{myprop}
