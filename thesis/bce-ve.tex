\chapter{Blocked Clause Elimination (BCE) and Variable Elimination (VE)}
\label{chap:bce-ve}

The authors of \cite{Jarvisalo_blockedclause} suggested that BCE
might improve on VE if performed after. In this appendix we will
study influence on performance of different combinations of BCE
(\ref{ssec:bce}) and VE (\ref{ssec:ve}).

The Figure \ref{fig:bce-ve} plots the following graphs:
\begin{itemize}
  \item \textsf{bce ve}: BCE performed before VE;
  \item \textsf{ve bce}: VE performed before BCE;
  \item \textsf{nobce ve}: VE is performed, BCE is not performed;
  \item \textsf{bce nove}: BCE is performed, VE is not performed;
  \item \textsf{nobce nove}: neither BCE nor VE are performed.
\end{itemize}

We observe that for nonrandom instances BCE performed before VE
is indeed slightly better than VE performed before BCE, but not
in any significant way.  However, from \textsf{bce ve} versus
\textsf{nobce ve} and \textsf{bce nove} versus \textsf{nobce nove}
we also observe that almost always BCE has a negative influence on
the overall performance.

For random instances it is better to use only VE, while for
nonrandom instances BCE should be performed after VE. In current
implementation, STRUCTure runs BCE before VE.

\begin{figure}
  \centering
  \plot{bce-ve}
  \caption{Effects of BCE and VE on solver's performance.}
  \label{fig:bce-ve}
\end{figure}
