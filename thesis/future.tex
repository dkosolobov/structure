\chapter{Future Work}
\label{chap:future}

While STRUCTure has most common simplification techniques included
is far from begin a competitive solver. This chapter lists some
ideas for future work that will improve STRUCTure's performance
and scalability.

The simplest idea to improve performance is to \emph{reduce the
data replication overhead}. In Figure \ref{fig:architecture}
every spawned activity takes input a formula and outputs a
smaller instance or a solution. To fit the Constellation model,
an activity is independent of other activities and, consequently,
it needs to build some internal data structure for reasonings it
performs. However, most internal data structures can be reused
by activity's children if they are executed on the same machine.
Profiling the solver showed that this overhead accounts for 10\%
to 40\% of the total running time not including the time garbage
collector spent freeing old structures.

Second idea is to exploit sequential points in the restart loop
and, possible, at the top of the search tree when not all threads
are busy. STRUCTure can be transformed into a \emph{portfolio
based parallel sat solver} \cite{5547119} by running other
specialized, possible incomplete, sequential solvers such as GSAT
\cite{Selman92anew} in parallel with the simplification component.



\todo{
Back Jumping.
Trim Learned Clauses.
Better Variable Selection.
testing combination of flags

Exploit sequential points
Reduce Data Replication and reuse internal data structures.
}
