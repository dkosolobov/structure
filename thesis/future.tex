\chapter{Future Work}
\label{chap:future}

While STRUCTure has most common simplification techniques included
it is far from being a competitive solver. This chapter lists some
ideas for future work that will improve STRUCTure's performance
and scalability.

The simplest idea to improve performance is to \emph{reduce the
data replication overhead}. In Figure \ref{fig:architecture}
every spawned activity takes input a formula and outputs a
smaller instance or a solution. To fit the Constellation model,
an activity is independent of other activities and, consequently,
it needs to build some internal data structure for reasonings it
performs. However, most internal data structures can be reused
by activity's children if they are executed on the same machine.
Profiling the solver showed that this overhead accounts for 10\%
to 40\% of the total running time not including the time garbage
collector spent freeing old structures.

Next, STRUCTure should exploit the idle threads available
during simplification. This can be done by running other
specialized, possible incomplete, sequential solvers such as GSAT
\cite{Selman92anew} in parallel with the simplification component.
Then STRUCTure will effectively become a \emph{portfolio based
parallel sat solver} \cite{5547119}.

Another thing to investigate is a better variable selection
heuristic.  Currently, STRUCTure uses a single variable selection
algorithm for all problem types, but, as we
have seen in chapter \ref{chap:results}, simplification procedures
have different performance impact on random and nonrandom instances.

Restarting could also benefit from a randomized model for
generations time to live instead of our simple increasing heuristic
\cite{Gomes:1998:BCS:295240.295710}.

\todo{Trim Learned Clause}

Finally, the effect of different parameters and flags needs to be
investigated. In chapter \ref{chap:results} STRUCTure was only
tested with a few simple combinations of options, but it may be
that certain combinations work better.
