\chapter{Future Work}
\label{chap:future}

While STRUCTure has most common simplification techniques included
it needs more work to become a competitive solver. This chapter
lists some ideas for future work that will improve STRUCTure's
performance and scalability.

The simplest idea to improve performance is to \emph{reduce the data
replication overhead}. In Figure \ref{fig:architecture} every spawned
activity takes as input a formula and outputs a smaller instance or a
solution. All internal data structures are build when the activity is
initalized and discarded when the activity finalizes. However, it is
possible to reuse some of these data structures to reduce overhead.

Next, STRUCTure should exploit the idle threads available
during simplification. This can be achieved by running other
specialized, possible stochastic, sequential solvers such as GSAT
\cite{Selman92anew} in parallel with the simplification component.
Then, STRUCTure will effectively have become a \emph{portfolio
based parallel sat solver} \cite{5547119}.

Another thing to investigate is a better branch selection heuristic.
Currently, STRUCTure uses one branch selection algorithm for all
problem types, but, as we have seen in Chapter \ref{chap:results},
simplification procedures have different performance impact on
random and nonrandom instances.

Restarting could also benefit from a randomized model for
generations time to live instead of our simple increasing heuristic
\cite{Gomes:1998:BCS:295240.295710}.

Finally, the effect of different combinations of parameters and
flags needs to be investigated. In Chapter \ref{chap:results}
STRUCTure was only tested with a few simple combinations of options,
but it may be that certain combinations work better.
