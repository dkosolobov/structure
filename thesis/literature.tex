\chapter{Related Work}
\label{chap:related}

SAT solvers are largely divided in two large categories:
\emph{stochastic} and \emph{complete}.  Stochastic solvers are
able to solve some type of problems very fast, but are not able
to prove unsatisfiability. An example of a stochastic algorithm
is GSAT \cite{Selman95localsearch} which initially assigns random
values to the Boolean variables and then arbitrarily flips the value
of a variable until a satisfying assignment is found or some time
has passed.

On the other hand, complete solvers must guarantee to find a
satisfying assignment or prove that no such assignment exists.
STRUCTure is such a solver. However, as complete solvers are
required to find only a single solution, they often make
assumptions to cut the searching space (see Pure Literal Rule,
Section \ref{ssec:pl}).  There are two common types of complete
solvers, both based on DPLL, a simple backtracking algorithm (see
Section \ref{ssec:dpll}): \emph{conflict-driven clause-learning}
(CD/CL) and \emph{look-ahead} solvers.

CD/CL solvers stem for the idea that conflicts (partial variable
assignments that cannot lead to a solution) are inevitable during
search \cite{Marques-silva99grasp:a}. The conflicts reached are
transformed into new clauses which reduce the searching space.
CD/CL solvers perform little reasoning during search and most
of the time is spent doing Boolean constraint propagation (BCP,
see Section \ref{ssec:bcp}) which makes it extremely important to
have an efficient BCP \cite{Moskewicz:2001:CEE:378239.379017}. Another
important step is restarting the search from time to time. CD/CL
solvers have proven to be very successful on industrial applications
of SAT where formulas are large, but have a small and easy core.

Look-ahead solvers perform additional reasoning during search in
order to minimize the number of variables assigned before a conflict
or a solution is reached. To decide on a branching variable
some variables are tested first to see which one reduces the formula
the most. This step is called a look ahead and, additionally,
it can learn new assignments. The core idea of look-ahead solvers is
that searching space can be reduced by branching less. Look-ahead
solvers perform better on hard small crafted and random instances.

STRUCTure is similar to look-ahead solvers in the sense that it
performs additional reasoning during search in order to branch less,
but the look-ahead step was replaced by Hyper-Unit Resolution (see
Section \ref{ssec:hbr}) and some heuristics for branching selection.
However, unlike other look-ahead solvers STRUCTure transforms
conflicts into clauses and adds them to the formula when the search
is restarted.

STRUCTure's approach to parallelism is based on the
divide-and-conquer principle. Branching divides the search space in
two and a solution in each space is searched independently. This
approach fits the Constellation programming model (see Section
\ref{sec:constellation}).

Parallel solvers that incorporate clause learning have different
mechanisms for spreading learned clauses among threads. The approach
in STRUCTure is to add the learned clauses after the search is
restarted. The need for synchronization is avoided by duplicating
learned clause at branching. The replication allows scalability at
the expense of greater overhead.

MiraXT \cite{mine:miraxt} is a parallel solver based on the
divide-and-conquer paradigm. Unlike STRUCTure, It maintains an
unique shared database of learned clauses. Access to the database
is synchronized.

GridSAT \cite{Chrabakh:2003:GCD:1048935.1050188} is a distributed
grid SAT solver using the master-slave model. Using the
divide-and-conquer principle the master divides the problem into
smaller independent subproblems solved by slaves in parallel. Learned
clauses are distributed only if they are smaller than a predefined
limit and the slaves incorporate them only after they backtrack to
the first decision level. In contrast to GridSAT, STRUCTure uses
a completely decentralized model and uses the divide-and-conquer
principle to solve the problem, not only to parallelize.

PSatz \cite{Jurkowiak_aparallelization} is a look-ahead solver
based on DPLL procedure that uses divide-and-conquer approach for
parallelization. Similar to Constellation, PSatz uses work stealing
for work balancing and distribution. Unlike STRUCTure, PSatz does
not have clause learning.

Another approach to parallelism are portfolio based SAT
solvers such as Plingeling \cite{mine:plingeling}, ManySat
\cite{Hamadi09manysat:a} or PPFolio \cite{mine:ppfolio} in which
several solvers or several instances of the same solver are run in
parallel. The computation stops when one solver finds a solution. The
idea is that different solvers perform better on different types
of problems. While portfolio based solvers perform exceptionally
on many problem types, with the increase of the number of available
cores they will still need to incorporate parallel solvers in order
to scale.

Manolios and Zhang use the power of Graphics Processing
Units to build a massive parallel, but incomplete, SAT
solver\cite{Manolios_implementingsurvey}. Their solution is an order
of magnitude faster than best known CPU based similar algorithms.

Finally, another simple solution to parallelism is to keep solver
sequential, but to exploit ability of common processors in order to execute
instructions that perform 32/64 boolean operations in parallel
\cite{mine:heule_parallel}.

