\documentclass[mathserif,serif]{beamer}

\usepackage{url}
\usepackage{amssymb}
\usepackage{color}
\usepackage{graphicx}
\usepackage{epstopdf}


\begin{document}

\title{STRUCTure: Towards a Distributed Boolean Satisfiability Solver}
\author{Alexandru Mosoi \\
        \url{ami650@few.vu.nl}}


\begin{frame}
  \titlepage

  \begin{flushright}
    Supervisor: Jason Maassen \url{jason@cs.vu.nl} \\
    Supervisor: Kees Verstoep \url{c.verstoep@vu.nl} 
  \end{flushright}
\end{frame}

\begin{frame}{Layout}
  \tableofcontents
\end{frame}

\section{Introduction}

\begin{frame}{Problem Definition}
  A \textcolor{red}{\textbf{Boolean formula}}, $F : \mathbb{B}^n \rightarrow
  \mathbb{B}$, is in Conjuctive Normal Form if
  it is a conjuction of disjunctions of literals.

  \begin{itemize}
    \item Variable: $1 \ldots N$
    \begin{itemize}
      \item Can take one of two truth values $True$ or $False$
    \end{itemize}
    \item Literal: $a$ or $\neg a$, $a$ variable
    \begin{itemize}
      \item $1, \neg 3$
    \end{itemize}
    \item Clause: $u_1 \lor u_2 \ldots \lor u_k$, $u_i$ literals
    \begin{itemize}
      \item $1 \lor 2 \lor \neg 4 \lor \neg 5 \lor 7$
    \end{itemize}
    \item Formula: $C_1 \land C_2 \ldots \land C_m$, $C_i$ clauses
    \begin{itemize}
      \item \textcolor{red}{$\mathbf{(1 \lor \neg 2) \land (1 \lor 3) \land (2 \lor 3 \lor 5)}$}
    \end{itemize}
    \item Any formula can be transformed such that clauses have at most 3
    literals ($3-SAT$ which is in NP).
  \end{itemize}
\end{frame}

\begin{frame}{Goal}
  \begin{block}{Goal}
    \centering \large
    Find a \textcolor{red}{satifying assignment} for any Boolean formula
    in CNF, or return that formula is inconsistent
  \end{block}

  \begin{block}{Example}
    \begin{itemize}
      \item $(1 \lor \neg 2) \land (1 \lor 3) \land (2 \lor 3 \lor 5)$
      \item 1 = False, 2 = False, 3 = True, 5 = False
      \item $\neg 1, \neg 2, 3, \neg 5$
    \end{itemize}
  \end{block}
\end{frame}

\begin{frame}{Importance}
  \begin{block}{Academia}
    \begin{itemize}
      \item One of Karp's 21 \textbf{NP-complete} problem
      \footnote{Richard M. Karp (1972). \emph{Reducibility Among Combinatorial Problems}. \url{http://www.cs.berkeley.edu/~luca/cs172/karp.pdf}}
      \item Other NP-complete problems can be reduced to $3-SAT$
      \begin{itemize}
        \item e.g: graph coloring, vertex cover
      \end{itemize}
    \end{itemize}
  \end{block}

  \begin{block}{Industry}
    \begin{itemize}
      \item Circuit encoding of Boolean logic
      \footnote{G. Audemard, L. Saıs. \emph{Circuit Based Encoding of CNF formula}}
      \begin{itemize}
        \item planning, synthesis, biology, verification, testing, routing, ...
      \end{itemize}
    \end{itemize}
  \end{block}
\end{frame}

\begin{frame}{Parallelization}
  \begin{itemize}
    \item State-of-art solvers are \emph{sequential}
    \begin{itemize}
      \item but they are wicked fast
    \end{itemize}
    \item Current approaches to parallelism:
    \begin{itemize}
      \item Run same solvers with different settings
      \begin{itemize}
        \item Pingeling, SAT4J \footnote{SAT4J. \url{http://www.sat4j.org/}}
      \end{itemize}
      \item Share database of learned clauses (usually limited to units and,
      maybe, binaries)
      \begin{itemize}
        \item Pingeling \footnote{Pingeling. \url{http://fmv.jku.at/papers/Biere-FMV-TR-10-1.pdf}}
        and Cryptominisat \footnote{Cryptominisat.  \url{http://www.msoos.org/cryptominisat2}}
      \end{itemize}
      \item Don't scale well
    \end{itemize}

    \item We want: \emph{Horizontal scalability}
    \footnote{\emph{Horizontal scalability} is Scalability in number of cpus}
    \begin{itemize}
      \item Replicated (not shared) memory
      \item Less synchronization
    \end{itemize}
  \end{itemize}
\end{frame}


\section{Architecture of STRUCTure}

\begin{frame}{Layout}
  \tableofcontents[currentsection]
\end{frame}

\begin{frame}{Types of solvers}
\let\thefootnote\relax\footnote{
Davis, Martin; Logemann, George, and Loveland, Donald (1962). \emph{A Machine Program for Theorem Proving}}

  \begin{columns}[T]
    \begin{column}[T]{0.4\textwidth}
      \centering
      \only<1>{\includegraphics[width=\textwidth]{data/solvers}}
      \only<2>{\includegraphics[width=\textwidth]{data/solvers-cdcl}}
      \only<3>{\includegraphics[width=\textwidth]{data/solvers-la}}
    \end{column}

    \begin{column}[T]{0.7\textwidth}
      \begin{itemize}
        \item Backtracking (DPLL)
        \only<1> {
          \begin{itemize}
            \item Pick a variable, $u$
            \item For both polarities $u, \neg u$:
            \begin{itemize}
              \item Simplify the formula
              \item If f. is empty return solution \\
              (empty formula is consistent)
              \item else if f. is inconsistent \textcolor{red}{backtrack} \\
              (found a \emph{conflict}) 
              \item else \textcolor{red}{recurse}
            \end{itemize}
          \end{itemize}
        }

        \item Conflict Driven/Clause Learning
        \only<2>{
          \begin{itemize}
            \item Searches conflicts
            \item \emph{Learns} new clauses to \emph{reduce} search space
            \item Maintain a database of learned clauses
          \end{itemize}
        }
        \item Look-ahead
        \only<3>{
          \begin{itemize}
            \item Picks variables which \emph{simplify} the formula
            as much as possible 
            \item Eventually the formula is \emph{empty} (consistent)
            or contains an \emph{empty clause} (inconsistent)
          \end{itemize}
        }
        \item<3> STRUCTure is Look-ahead + Clause Learning
      \end{itemize}
    \end{column}
  \end{columns}
\end{frame}


\begin{frame}{Constellation}
  \let\thefootnote\relax\footnote{J. Maassen, F.J. Seinstra, H.E. Bal,
  \emph{Context Aware Many-Task Computing with Ibis/Constellation}}

  \begin{columns}[t]
    \begin{column}[T]{0.3\textwidth}
      \centering
      \includegraphics[width=\textwidth]{data/constellation}
    \end{column}

    \begin{column}[T]{0.7\textwidth}
      \begin{itemize}
        \only<1> {
          \item Computational model based on activities which:
          \begin{itemize}
            \item perform work
            \item spawn new activities
            \item send/process events to/from other activities
          \end{itemize}
        }

        \only<2> {
          \item Each node (machine) has a fixed number of executors
          (threads) which run activities.
          \item \emph{Work stealing}
          \begin{itemize}
            \item Local: \emph{small jobs}
            \item Remote: \emph{large jobs}
          \end{itemize}
        }

        \only<3> {
          \item Advantage:
          \begin{itemize}
            \item Easy to parallelize (no synchronization)
            \item Scalable
          \end{itemize}
          \item Disadvantage:
          \begin{itemize}
            \item No \emph{shared memory} or \emph{broadcasting}
            \item \emph{No cancel}
          \end{itemize}
        }
      \end{itemize}
    \end{column}
  \end{columns}
\end{frame}

\begin{frame}{Architecture of STRUCTure}
  \centering
  \includegraphics[width=0.8\textwidth]{data/activity}
\end{frame}

\begin{frame}{XOR Gates and Dependent Variable Removal
\footnote{M.J.H. Heule. \emph{Towards a lookahead sat solver for general purposes.}}
\footnote{J.A. Roy, I.L. Markov. \emph{Restoring circuit structure from sat instances}.}}

  \begin{columns}[t]
    \begin{column}[T]{0.3\textwidth}
      \includegraphics[width=\textwidth]{data/xordvr}
    \end{column}

    \begin{column}[T]{0.8\textwidth}
      \begin{itemize}
        \item Many instances encode several logical gates: \emph{OR, AND, XOR}.
        \item \emph{XOR} with $N$ inputs: \textcolor{red}{$\mathbf{2^{N}}$} CNF
        clauses of \textcolor{red}{$N + 1$} literals
      \end{itemize}
    \end{column}
  \end{columns}

  \begin{block}{Idea}
    \emph{Decode} XOR gates from formula and use them as eqns. in $\mathbb{Z}_2$.
  \end{block}

  \begin{block}{Dependent Variable Removal}
    \begin{itemize}
      \item Variable is dependent if it appears \emph{only} in XOR gates.
      \item Can be removed together with some clauses.
    \end{itemize}
  \end{block}
\end{frame}

\begin{frame}{Restart Loop}
  \begin{center}
    \includegraphics[width=0.6\textwidth]{data/restart}
  \end{center}

  \begin{block}{Idea}
    Restart computation, maybe a different variable selection order
    leads faster to a solution.
  \end{block}

  \begin{block}{How}
    \begin{itemize}
      \item Simplifies the formula
      \only<1>{
        \begin{itemize}
          \item Blocked Clause Elimination
          \footnote{M Jarvisalo, A. Biere, M. Heule, \emph{Blocked Clause Elimination}}
          \item Variable Elimination
          \item Simplify
        \end{itemize}
      }
      \only<2>{
        \item \textcolor{red}{Starts} and \textcolor{red}{stops} searching after some time
        \item Extend formula with \textcolor{red}{learned clauses}
      }
    \end{itemize}
  \end{block}

\end{frame}

\begin{frame}{Searching}
  \begin{columns}[t]
    \begin{column}[T]{0.3\textwidth}
      \centering
      \includegraphics[width=\textwidth]{data/seaching}
    \end{column}

    \begin{column}[T]{0.8\textwidth}
      \begin{itemize}
        \item \emph{Split} divides formula into a disjunction of 
        independent smaller formulas
        \item \emph{Select} picks a variable for branching
        \begin{itemize}
          \item analysis of clauses in which the variables appear
        \end{itemize}
        \item \emph{Branch} is the backtracking step
        \item \emph{BlackHole} filters out instances from old restarts
        \item \emph{Solve} simplifies the formula with newest added branch
      \end{itemize}
    \end{column}
  \end{columns}
\end{frame}

\begin{frame}{Learning}
  \begin{columns}[t]
    \begin{column}[T]{0.5\textwidth}
      \centering
      \includegraphics[width=\textwidth]{data/learning}
    \end{column}

    \begin{column}[T]{0.6\textwidth}
      \begin{itemize}
        \item \emph{Conflict}: cannot assign selected literals \textcolor{red}{$1, 2, \neg 4$}
        \item $\overline{(1 \land 2 \land \neg 4)} \equiv (\neg 1 \lor \neg 2 \lor 4)$
        \item $(\neg 1 \lor 2), (1 \lor 3)$
        \item Next time no need to search the same space again.
        \item Assign $\neg 1$ then $3$ must be false
      \end{itemize}
    \end{column}
  \end{columns}
\end{frame}

\begin{frame}{Look-ahead}
  \begin{columns}[t]
    \begin{column}[T]{0.4\textwidth}
      \includegraphics[width=\textwidth]{data/lookahead}
    \end{column}

    \begin{column}[T]{0.7\textwidth}
      \includegraphics[width=\textwidth]{data/lookahead-tree}
    \end{column}
  \end{columns}

  \begin{block}{Idea}
    \begin{itemize}
      \item Perform search from root until depth 1
      \item \emph{Learn units} and sometimes binaries
      \begin{itemize}
        \item unit = clause of length 1 = variable assignment
      \end{itemize}
      \item Then solve problem with newest clauses
    \end{itemize}
  \end{block}

\end{frame}


\section{Experimental Results}

\begin{frame}{Layout}
  \tableofcontents[currentsection]
\end{frame}

\begin{frame}{Instances}
    \begin{itemize}
      \item 678 instances taken from SAT Competition 2009
      \item \textbf{170 instances} solvable by STRUCTure
      \begin{itemize}
        \item Sequential champions go to about 250
      \end{itemize}
      \item Timelimit: \textbf{15 minutes}
      \begin{itemize}
        \item SAT Competition limit is 20 minutes
        \item Enough to understand performance
      \end{itemize}

      \item Evaluation was done on DAS-4 cluster at VU
      \url{http://www.cs.vu.nl/das4/}
    \end{itemize}
\end{frame}

\begin{frame}{1 to 8 CPUs (1 node)}
  \begin{itemize}
    \item using a subset of \textbf{175 instances}
  \end{itemize}
  \centering
  \includegraphics[width=0.6\textwidth,angle=-90]{data/1machine}
\end{frame}

\begin{frame}{1 to 8 nodes (1 CPU)}
  \begin{itemize}
    \item using a subset of \textbf{175 instances}
  \end{itemize}
  \centering
  \includegraphics[width=0.6\textwidth,angle=-90]{data/1cpu}
\end{frame}

\begin{frame}{Superimposed}
  \begin{itemize}
    \item using a subset of \textbf{175 instances}
  \end{itemize}
  \centering
  \includegraphics[width=0.6\textwidth,angle=-90]{data/superimposed}
\end{frame}

\begin{frame}{1 to 8 nodes (8 CPUs)}
  \begin{itemize}
    \item using a subset of \textbf{175 instances}
  \end{itemize}
  \centering
  \includegraphics[width=0.6\textwidth,angle=-90]{data/8cpu}
\end{frame}

\begin{frame}{Disabling various strategies}
  \includegraphics[width=0.6\textwidth,angle=-90]{data/nc}
\end{frame}

\begin{frame}{Selected instances}
  \small

  \begin{tabular}{l | c c | c c c c}
    & & & \multicolumn{4}{c}{Workers (Time:\textcolor{red}{speedup})} \\
    \hline
    Instance & & & 1 & 2 & 4 & 8 \\
    \hline
    countbitsrotate016 & U & 1n & 666:\textcolor{red}{1} & 363:\textcolor{red}{1.8} & 189:\textcolor{red}{3.5} & 105:\textcolor{red}{6.3} \\
                              & & 1c/n & 666:\textcolor{red}{1} & 397:\textcolor{red}{1.7} & 247:\textcolor{red}{2.7} & 191:\textcolor{red}{3.5} \\
    \hline
    AProVE09-19 & S & 1n & 129:\textcolor{red}{1} & 65:\textcolor{red}{2.0} & 29:\textcolor{red}{4.4} & 20:\textcolor{red}{6.5} \\
                & & 1c/n & 129:\textcolor{red}{1} & 117:\textcolor{red}{1.1} & 70:\textcolor{red}{1.8} & 62:\textcolor{red}{2.1} \\
    \hline
    \hline
    instance\_n5\_i6\_pp\_ci\_ce & U & 1n & 242:\textcolor{red}{1} & 94:\textcolor{red}{2.6} & 59:\textcolor{red}{4.1} & 36:\textcolor{red}{6.7}  \\
                                 & & 1c/n & 242:\textcolor{red}{1} & 114:\textcolor{red}{2.1} & 116:\textcolor{red}{2.1} & 99:\textcolor{red}{2.4} \\
    \hline
    instance\_n5\_i5\_pp & S & 1n & 102:\textcolor{red}{1} & 65:\textcolor{red}{1.6} & 49:\textcolor{red}{2.1} & 32:\textcolor{red}{3.2} \\
                         & & 1c/n & 102:\textcolor{red}{1} & 61:\textcolor{red}{1.7} & 52:\textcolor{red}{2.0} & 30:\textcolor{red}{3.4} \\
    \hline
    \hline
    unif-k3-$\ldots$ & U & 1n & 1886:\textcolor{red}{1} & 951:\textcolor{red}{2.0} & 484:\textcolor{red}{3.9} & 245:\textcolor{red}{7.7} \\
                       & & 1c/n & 1886:\textcolor{red}{1} & 992:\textcolor{red}{1.9} & 589:\textcolor{red}{3.2} & 375:\textcolor{red}{5.0} \\
    \hline
    unif-k3-$\ldots$ & S & 1n & 237:\textcolor{red}{1} & 91:\textcolor{red}{2.6} & 43:\textcolor{red}{5.5} & 13:\textcolor{red}{18.2} \\
                       & & 1c/n & 237:\textcolor{red}{1} & 51:\textcolor{red}{4.6} & 68:\textcolor{red}{3.5} & 24:\textcolor{red}{9.9} \\
    \hline
    \hline
  \end{tabular}

  \bigskip
  S = Satisfiable, U = Unsatisfiable \\
  1n = 1 node, \# cpus varies \\
  1c/n = 1 cpu / node, \# nodes varies \\
\end{frame}

\section{Future work}

\begin{frame}{Future work}
  \begin{itemize}
    \item Find a better variable selection algorithm    
    \item Remove learned clauses
    \item Reduce the amount of data replication
    \begin{itemize}
      \item current \emph{overhead} from 10\% on big instances, to 40\% on small hard instances
    \end{itemize}
  \end{itemize}
\end{frame}

\section{Conclusions}

\begin{frame}{Conclusions}
  \begin{itemize}
    \item I built a sat solver with horizontal scalability
    \begin{itemize}
      \item 1 node/1 core performance sacrificed
    \end{itemize}
    \item with good distributed learning.
    \item \emph{ManySat} type solvers
    \footnote{ManySat run multiple solvers in parallel}
    \begin{itemize}
      \item \emph{don't scale}
      \item good performance because of \emph{specialization}
    \end{itemize}
    \item Constellation makes parallelization very easy
    \begin{itemize}
      \item As long as your program fits Constellation model, but
      \item \emph{No cancellation} can be a problem when you want to stop
      \item \emph{Process event and wait} model creates inefficiencies.
    \end{itemize}
  \end{itemize}
\end{frame}

\begin{frame}{Questions?}
  \begin{block}{Contact}
    \begin{itemize}
      \item Can be downloaded from \\ \url{https://github.com/brtzsnr/structure}
      \item I can be contacted at \\ \url{brtzsnr@gmail.com}
    \end{itemize}
  \end{block}


  %\begin{block}{Thank you}
    %\begin{itemize}
      %\item Kees Verstoep for offering me the position of
      %student assitantship in autumn 2009 to work on sat solving.
    %\end{itemize}
  %\end{block}
\end{frame}


\begin{frame}{XOR Gates and Dependent Variable Removal
\footnote{M.J.H. Heule. \emph{Towards a lookahead sat solver for general purposes.}}
\footnote{J.A. Roy, I.L. Markov. \emph{Restoring circuit structure from sat instances}.}}

  \begin{columns}[t]
    \begin{column}[T]{0.3\textwidth}
      \includegraphics[width=\textwidth]{data/xordvr}
    \end{column}

    \begin{column}[T]{0.8\textwidth}
      \begin{itemize}
        \item Many instances encode several logical gates: \emph{OR, AND, XOR}.
        \item \emph{XOR} with $N$ inputs: \textcolor{red}{$\mathbf{2^{N}}$} CNF
        clauses of \textcolor{red}{$N + 1$} literals
      \end{itemize}
    \end{column}
  \end{columns}

  \begin{block}{Idea}
    \emph{Decode} XOR gates from formula and use them as eqns. in $\mathbb{Z}_2$.
  \end{block}

  \begin{block}{Dependent Variable Removal}
    \begin{itemize}
      \item If a variable appears in exactly \emph{one XOR gate} then the XOR gate
      \emph{can be removed} from the formula
        \begin{itemize}
          \item because the variable can always be set to satisfy
        \end{itemize}
      \item Gaussian Elimination used to get more dependent vars
      \begin{itemize}
        \item Pick a variable in a gate and knock it out from other gates.
      \end{itemize}
    \end{itemize}
  \end{block}
\end{frame}


\begin{frame}{Dependendent Variable Removal}
  \begin{table}
    \framebox{
      \begin{tabular}{l c c}
        Instance
        \footnote{SAT07/crafted/Difficult/contest05/jarvisalo/mod2-rand3bip-sat-$\ldots$.reshuffled-07.cnf}
        \footnote{SAT07/crafted/Medium/contest05/jarvisalo/mod2-rand3bip-sat-$\ldots$.reshuffled-07.cnf}
        
        & CryptoMinisat 2.6 & STRUCTure \\
        \hline
        $\ldots$270-2.sat05-2249$\ldots$ & - & 0.319 \\
        $\ldots$270-1.sat05-2248$\ldots$ & - & 0.306 \\
        $\ldots$250-3.sat05-2220$\ldots$ & - & 0.296 \\
        $\ldots$230-2.sat05-2189$\ldots$ & 497.877 & 0.291 \\
        $\ldots$280-1.sat05-2263$\ldots$ & 272.806 & 0.286 \\
        $\ldots$280-3.sat05-2265$\ldots$ & - & 0.283 \\
        $\ldots$280-2.sat05-2264$\ldots$ & - & 0.281 \\
        $\ldots$210-2.sat05-2159$\ldots$ & 50.196 & 0.246 \\
        $\ldots$220-1.sat05-2173$\ldots$ & 44.195 & 0.231
      \end{tabular}
    }
    \caption{Times on some instances containing many Dependent Variables}
  \end{table}
\end{frame}


\end{document}
