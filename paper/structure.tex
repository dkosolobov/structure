\documentclass[12pt]{article}

\usepackage{amsmath}
\usepackage{amsthm}
\usepackage{algorithm}
\usepackage{algorithmic}
\usepackage{fullpage}

\newtheorem{definition}{Definition}
\newtheorem{corollary}{Corollary}

\newcommand{\T}{\text{TRUE}}
\newcommand{\F}{\text{FALSE}}

\renewcommand{\algorithmiccomment}[1]{// #1}


\begin{document}

\title{STRUCTure: A parallel boolean satifibiality solver}
\author{Alexandru Moșoi $<$brtzsnr@gmail.com$>$}
\maketitle

\begin{abstract}
\end{abstract}

\section{Introduction}

The \emph{satisfiability problem} (\emph{SAT}) is a decision problem whose instance is a boolean
expression using only AND ($\land$), OR ($\lor$), NEGATION ($\neg$), boolean variables and
parenthesis. The task is to find an assignment of TRUE and FALSE to the variables such that
the instance is satisfied. A formula is \emph{satisfiable} if such an assignment exists and
\emph{unsatisfiable} otherwise.

STRUCTure is a parallel sat solver based on Constellation framework. The Constellation's
parallel programming model is based on job spawning and message passing among jobs.  A job can
be executed on any machine. There is no concept of shared memory in Constellation which deals
aways of consistency problems.

A \emph{literal} is a positive or a negative variable. A \emph{clause} is disjunction of literals.
A \emph{formula} is in conjunctive normal form (CNF) if it is a conjunction of clauses. For
example the CNF formula
\[
x_1 \land (\neg x_1 \lor \neg x_2) \land (x_2 \lor x_3 \lor x_4)
\]
is satisfied for $x_1 = \T$, $x_2 = \F$ and $x_3 = \text{TRUE}$. A formula is in \emph{k-SAT}
if every clause in the formula contains at most $k$ literals. \emph{1-SAT} has a trivial solution
and \emph{2-SAT} can be solved in polynomial time.

\begin{definition}
An \emph{empty formula} is a formula containing zero clauses.
\end{definition}

\begin{corollary}
An \emph{empty formula} is always satisfied.
\end{corollary}

\begin{definition}
An \emph{empty clause} is a clause of length 0.
\end{definition}

\begin{corollary}
An \emph{empty clause} cannot be satisfied and therefore a formula containing an empty clause
is unsatisfiable.
\end{corollary}

\begin{definition}
A \emph{unit clause} is a clause of length 1.
\end{definition}
If a formula contains an unit clause the satisfying
assignment will satisfy the unit clause, too. Therefore, the assignment $x_1 = \text{TRUE}$
is equivalent to unit clause $x_1$ and the assignment $x_1 = \text{FALSE}$ is equivalent to
unit clause $\neg x_1$.

\begin{definition}
A \emph{binary clause} is a clause of length 2.
\end{definition}

\begin{corollary}
$u \lor v \iff \neg u \rightarrow v \iff \neg v \rightarrow u$. A binary clause is
equivalent to an implication relation.
\end{corollary}

The implication relation has the following properties:
\begin{itemize}
\item \emph{reflexive}: $u \rightarrow u$
\item \emph{asymmetric}: $u \rightarrow v \implies \neg u \rightarrow \neg v$
\item \emph{transitive}: $(u \rightarrow v) \land (v \rightarrow t) \implies u \rightarrow t$
\end{itemize}


\section{Reasoning}

The goal of reasoning in a SAT solver is to reduce a formula $F$ to a simpler formula $F'$ such
that an assignment $\alpha$ for $F'$ is valid for $F$ too. The common types of simplifications
are removing redundant literals and clauses or adding new ones to strengthen the formula.


\subsection{Basic Simplifications}

\begin{corollary}
$F \land \T \iff F$. A satisfied clause can be removed from the formula preserving satisfiability.
\end{corollary}

\begin{corollary}
$F \land \F \iff \F$. A falsified clause falsifies the formula.
\end{corollary}

\begin{corollary}
$(u \lor u \lor x_1 \lor \ldots \lor x_k) \iff (u \lor x_1 \lor \ldots \lor x_k)$.
A clause containing duplicate literals is equivalent to a clause containing the same literals,
but with duplicates removed.
\end{corollary}

\begin{corollary}
$(u \lor \neg u \lor x_1 \lor \ldots \lor x_k) \iff \T$.
A clause containing a variable and its negation is a tautology.
\end{corollary}

\begin{corollary}
$u \land \neg u \iff \F$. A variable cannot take both true and false values simultaneously.
\end{corollary}

\begin{corollary}[Resolution]
$(u \lor a_1 \lor \ldots \lor a_{k_a}) \land (-u \lor b_1 \lor \ldots \lor b_{k_b}) \implies (a_1 \lor \ldots \lor a_{k_a} \lor b_1 \lor \ldots \lor b_{k_b})$
\end{corollary}

\begin{corollary}[Binary resolution]
\label{cor:binary-resolution}
$(u \rightarrow v) \land (\neg u \lor \neg v \lor x_1 \lor \ldots \lor x_k) \implies (\neg u \lor x_1 \lor \ldots \lor x_k)$
\end{corollary}


\subsection{Implication Graph}

\begin{definition}
The \emph{implication graph} of a formula $F$ is an oriented graph $G = (V, E)$ where $V$ is the
set of all literals and $E = \{(u, v) \in V \times V | u \rightarrow v\}$.
\end{definition}

\begin{corollary}
$u \rightarrow v$ iff there is a path from $u$ to $v$ in graph $G$.
\end{corollary}

\begin{corollary}
$(u \rightarrow v) \land (v \rightarrow u) \implies (u \leftrightarrow v)$.
\end{corollary}

If $u \leftrightarrow v$ then every occurrence of $v$ can be replaced with $v$. More generally,
in the implication graph each strongly connected component can be collapsed to a single literal
and, in the formula, every literal from the component can be replaced with the destination literal.

\begin{corollary}[Contradiction]
$u \rightarrow \neg u \implies \neg u$
\end{corollary}

In STRUCTure the implication graph is maintained as a directed acyclic graph. If a contradiction
is discovered, solved units are propagated. If a new binary (e.g. from hyper-binary contradiction)
creates a cycle then the cycle is collapsed and literals are renamed in the formula.


\subsection{Unit Propagation}
\begin{definition}
Given a formula $F$ and a literal $u$ \emph{unit propagation}, denoted $UP(F, u)$, simplifies
F according to two rules:
\begin{enumerate}
\item $\neg u$ is removed from every clause containing it
\item every clause containing literal $u$ is removed
\end{enumerate}
\end{definition}

The resulted formula is equivalent with the old one given the literal and can contain other
literals that in turn can be propagated.


\subsection{Hyper Resolution}

Hyper resolution is realized by applying binary resolution (corollary \ref{cor:binary-resolution})
repeatedly.

\begin{corollary}[Hyper-unary resolution]
$(u \rightarrow \neg x_1) \land \ldots \land (u \rightarrow \neg x_n) \land (x_1 \lor \dots \lor x_n) \implies \neg u$
\end{corollary}

\begin{corollary}[Hyper-binary resolution]
$(u \rightarrow \neg x_1) \land \ldots \land (u \rightarrow \neg x_n) \land (a \lor x_1 \lor \dots \lor x_n) \implies (\neg u \lor a)$
\end{corollary}

\begin{corollary}[Hyper-ternary resolution]
$(u \rightarrow \neg x_1) \land \ldots \land (u \rightarrow \neg x_n) \land (a \lor b \lor x_1 \lor \dots \lor x_n) \implies (\neg u \lor a \lor b)$
\end{corollary}

The algorithm \ref{alg:hyper-resolution} applies hyper-\{unary, binary, ternary\} on a CNF
formula. In practice, many binary clauses generated are redundant (e.g. two consecutive runs
of hyper-resolution will generate almost the same clauses) and ternary clauses are used only
to find extra unary and binary clauses through resolution. \footnote{In STRUCTure hyper-ternary
is under development}.

\begin{algorithm}[h!]
\begin{algorithmic}
\REQUIRE{$F$ a CNF formula}
\ENSURE{new units, binaries and ternaries}
\STATE $\Sigma_0, \Sigma_0' \gets \{\}, 0$
\STATE $\Sigma_1, \Sigma_1' \gets \{\}, 0$
\STATE $\Sigma_2, \Sigma_2' \gets \{\}, 0$
\FOR[for every clause in formula]{$C \in F$}
  \FOR[for every literal in clause]{$u \in C$}
    \STATE $\Sigma_0' \gets \Sigma_0' + u^0$
    \STATE $\Sigma_1' \gets \Sigma_1' + u^1$
    \STATE $\Sigma_2' \gets \Sigma_2' + u^2$
    \FORALL[for every neighbour of literal]{$v$ s.t. $v \rightarrow \neg u$}
      \STATE $\Sigma_0[v] \gets \Sigma_0[v] + u^0$
      \STATE $\Sigma_1[v] \gets \Sigma_1[v] + u^1$
      \STATE $\Sigma_2[v] \gets \Sigma_2[v] + u^2$
    \ENDFOR
  \ENDFOR
  \FORALL{literals $u$}
    \IF[Hyper-unary resolution]{$\Sigma_0[u] = \Sigma_0'$}
      \PRINT{new unit $u$}
    \ELSIF[Hyper-binary resolution]{$\Sigma_0[u] = \Sigma_0' - 1$}
      \STATE $v \gets \Sigma_1' - \Sigma_1[u]$
      \PRINT{new binary $u \lor v$}
    \ELSIF[Hyper-ternary resolution]{$\Sigma_0[u] = \Sigma_0' - 1$}
      \STATE \COMMENT{The following system of equations results}
      \STATE $v + t = \Sigma_1' - \Sigma_1[u]$
      \STATE $v^2 + t^2 = \Sigma_2' - \Sigma_2[u]$
      \PRINT{new ternary $u \lor v \lor t$}
    \ENDIF
  \ENDFOR
\ENDFOR
\end{algorithmic}

\caption{Hyper resolution}
\label{alg:hyper-resolution}
\end{algorithm}


\subsection{Pure Literals}

\begin{definition}
In a formula $F$ a literal $l$ is said to be \emph{pure} if the negated literal $\neg l$ does
not appear in the formula.
\end{definition}

\begin{corollary}
All pure literals can be assigned preserving the satisfiability of the formula.
\end{corollary}

\subsection{(Self-)Sub Summing}

\begin{corollary}[Sub Summing]
$\begin{array}{rl}
x_1 \lor \ldots \lor x_i &\lor x_{i+1} \lor \ldots \lor x_n \\
x_1 \lor \ldots \lor x_i & \\
\hline
x_1 \lor \ldots \lor x_i &
\end{array}$. If a clause includes another clause then the former is satisfied whenever the
later is satisfied.
\end{corollary}

All clauses that include other clauses can be removed from a formula preserving satisfiability.

\begin{corollary}[Self-sub suming \footnote{Self-sub summing is in progress}]
$\begin{array}{rl}
\neg u \lor x_1 \lor \ldots \lor x_i &\lor x_{i+1} \lor \ldots \lor x_n \\
u \lor x_1 \lor \ldots \lor x_i & \\
\hline
x_1 \lor \ldots \lor x_i &\lor x_{i+1} \lor \ldots \lor x_n \\
u \lor x_1 \lor \ldots \lor x_i &
\end{array}$
\end{corollary}


\section{Solver}

The core solver is composed of two parts: simplification (algorithm \ref{alg:simplify}) and
branching (algorithm \ref{alg:solve}).

\begin{algorithm}[h!]
\begin{algorithmic}
\REQUIRE{$F$ formula}
\ENSURE{a simplified formula}

\WHILE{simplification possible}
  \STATE propagate units
  \STATE hyper-binary resolution
\ENDWHILE

\STATE (self-)sub summing
\STATE propagate pure literals
\STATE solve if a trivial solution exists (e.g. formula is 2-SAT)

\end{algorithmic}
\caption{Simplify}
\label{alg:simplify}
\end{algorithm}

\begin{algorithm}[h!]
\begin{algorithmic}
\REQUIRE{$F$ formula, $l$ branch}
\ENSURE{a satisfying assignment for $F$}

\STATE $solution, F' \gets \textsf{Simplify}(F \land l)$
\IF{$solution$ = satisfied}
  \RETURN satisfied, $F'$
\ELSIF{$solution$ = unsatisfied}
  \RETURN unsatisfied
\ELSE[unknown solution]
  \STATE $branch \gets \textsf{SelectBranch}(F')$
  \STATE async $positiveSolved, positiveSolution \gets \textsf{Solve}(F', branch)$
  \STATE async $negativeSolved, negativeSolution \gets \textsf{Solve}(F', \neg branch)$
  \IF{$positiveSolved = \textsf{satified}$}
    \RETURN satisfied, $positiveSolution$
  \ELSIF{$negativeSolved = \textsf{satified}$}
    \RETURN satisfied, $negativeSolution$
  \ELSE
    \RETURN unsatisfied
  \ENDIF
\ENDIF
\end{algorithmic}
\caption{Solve}
\label{alg:solve}
\end{algorithm}


\section{Future Work}


\subsection{Simplification}

The power of the solver lies in the ability to simplify the formula before branching.

\emph{Blocked clause elimination} is a technique (more powerful than pure literals) to remove some
variables which are proved to be dependent on other variable. This technique also helps with XOR
clauses which result from circuit encodings and which are not usually handle well by SAT solvers.

\emph{Self-sub summing} and \emph{hyper-ternary resolution} are in progress. The problem is
finding efficient algorithms, otherwise the performance cost will offset the benefits of
not branching.


\subsection{Clause sharing}

Many conflict-driven sat solvers benefit from clause learning and sharing which reduce the search
space. Clause learning in STRUCTure is possible when walking up the branch tree.  However clause
sharing in a distributed system is harder without shared memory. Constellation's memory model is
based on message passing.  One idea to make clause sharing possible is to do a iterative-deepening
backtracking. At each new maximum depth learned clauses are distributed down the branching tree.


\end{document}
